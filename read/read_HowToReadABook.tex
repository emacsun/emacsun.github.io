\documentclass[10pt,a4paper,UTF8]{article}
\usepackage{zclorg}
\author{Eastern(ZCL)}
\date{}
\title{主动阅读 --《如何阅读一本书》}
\hypersetup{
 pdfauthor={Eastern(ZCL)},
 pdftitle={主动阅读 --《如何阅读一本书》},
 pdfkeywords={reading},
 pdfsubject={这是《如何阅读一本书》的读后感,然而有不完全受限于此书。本文第一次对我的知识获取过程进行了全面的批判性的分析,反思了以往学习方式中存在的缺陷,明确了自己的学习流程。},
 pdfcreator={Emacs 25.0.50.1 (Org mode 8.3.2)}, 
 pdflang={English}}
\begin{document}

\maketitle
\tableofcontents


\section{引言}
\label{sec:orgheadline1}


“半亩方塘一鉴开,天光云影共徘徊。问渠哪得清如许?为有源头活水来。” 作为知识的源头活水,书籍在人类进步的道路上起到了举足轻重的作用。然而,从书籍到知识这一过程却并不是那么一帆风顺。读书,绝对是一门艺术。会读书,事半功倍;不会,则事倍功半。

从书本中获得知识不同于从影视材料中获得知识,总体而言影视材料是对原著的一次加工和筛选。从信息理论而言,根据香农信息理论,经过处理的材料其携带的信息只会减少不会增多。从个人阅读体验来讲,影视材料也限制了个人思考。另外,从书本中获得知识也不同于有老师指导的知识获得,如果你问一个老师问题,他可能会给你答案。如果你仍然不明白,你还可以追问,省下自己的时间。然而,当你问一本书一个问题,你就必须自己回答这个问题。在这样的情况下,这本书就与自然或者大千世界一样。当你提出问题时,只有等你自己综合书中的材料,甚至需要查阅其他资料,才可以豁然开朗,拨云见日。影视材料和老师的指导是双刃剑,一方面他引导了思考,但是又剥夺了思考的机会,而读书这完全赋予读者独立思考的机会。

主动阅读需要的不仅仅是思考,一个人还需要运用他的感觉与想象力。在阅读过程中,一个人必须观察,记忆,在看不到的地方运用想象力。所以通过主动阅读一个人可期望拥有敏锐的观察,灵敏可靠的记忆以及想象的能力。当然,最重要的是训练有素的分析反思能力。归根结底,主动阅读是一个自我发现的过程。

《如何阅读一本书》是一本不可多得的介绍读书方法论的好书,适合反复咀嚼,仔细品味。至少,于我,这本书介绍的主动阅读涉及的方法彻底改变了我的读书流程,颠覆了我的“书观”。需要指出的是,并不是所有的材料都适合我们按照主动阅读的原则来分析,很多时候为了快速的了解一个领域,有影视材料和老师的辅助往往更加有效率。正所谓“师傅领进门,修行靠个人”。主动阅读就是一种修行。

\section{主动阅读}
\label{sec:orgheadline9}


“主动阅读”是《如何阅读一本书》的核心。“主动阅读”,顾名思义不同于被动阅读,是作者主动从书中获取自己想要的知识而不是被动的输入书中的文字。主动阅读时一个探索的过程,被动阅读仅仅是一个输入的过程,最多达到对书中材料的记忆,却无法将书中的知识嵌入自己已有的知识体系。主动阅读达到的目的是对知识的调用而不是对知识的记忆。

在《如何阅读一本书》中,作者通过四个层次实现主动阅读:1)基础阅读,2)检视阅读,3)分析阅读和4)主题阅读来完成。这四个层次是渐近的,较高的层次包含较低的层次。检视阅读就包括了基础阅读,以此类推,主题阅读包含了前三层次的阅读。
\subsection{基础阅读}
\label{sec:orgheadline2}


基础阅读又叫初步阅读,简单来说只要只要一个人熟练了这个层次的阅读,就基本摆脱文盲的状态。也可以反过来表述,只要一个人不是文盲,他就可以胜任基础阅读。对于基础阅读的学习是在小学的时候完成的,那个时候还处于识字阶段,字典是最好的帮手。然而,当一个读者开始阅读非母语的资料时,免不了仍然要从基础阅读开始。或者,对于中文而言,阅读白话文很容易就完成基础阅读,但是对于文言文一切可能没有那么顺利。基础阅读是开展其他层次阅读的基础。显然,你识得这篇文章的每一个字,你就达到了对这篇文章基础阅读的层次。
\subsection{检视阅读}
\label{sec:orgheadline3}


检视阅读是我经常忽略的一个细节,通过略读来达到检视阅读的目的。往往,我是略读了,却没有达到检视阅读的层次。因为我根本不知道在检视阅读的过程中需要解决那些问题,也没有什么方法论可以遵循。所以,通常情况是,在拿到一本书后,我会前翻翻后翻翻,看看封皮看看目录,最多看看各个章节都介绍了什么东西。粗看,我的确进行了检视阅读,实则不然,在这一步我没有回答任何问题,也不知道要找什么答案。有问题导向的学习才是高效的学习(请见我的博文《我的知识流》)。在我阅读《如何阅读一本书》之前,我的检视阅读是做的不够充分。

总体而言在阅读一本书的过程中,一个有追求和修养的阅读者始终要努力回答一下四个问题:
\begin{enumerate}
\item \textbf{作者谈了什么?} 要能够用自己的语言对此进行概括。作者的核心主题是什么?

\item \textbf{作者是如何展开核心主题的?} 作者往往会把一个核心主题分成几个部分,并且在这些部分中也各有自己的中心。

\item \textbf{这本书说的有道理么?} 部分有道理还是全部有道理?除非回答了前两个问题,否则不可能对这个问题有什么见解。读完一本书并对此有个基本的判断才是主动阅读者该做的事情。

\item \textbf{这本书跟我有什么关系?} 这本书与我已有的知识体系之间有什么联系?调用已有的知识并与书中的知识发生链接,这是我们阅读的最终目的。只有通过这种方式构建起来的知识体系才构成一个整体,不至于一盘散沙。
\end{enumerate}

检视阅读可以帮助我们初步的回答前两个问题,通过快速的浏览整书就可以把握书中知识体系的结构。值得注意的是,在检视阅读的过程中强调一个快字,不需要在细枝末节上逗留。因为初步的回答前两个问题也不需要我们在细节上投入过多时间。很多时候,我拿到一本跟数学有较强关联的书籍,比如信息论,信道编解码方面的书籍,在一开始就投入过多精力,结果只见树木不见森林,而且过多的细节追究会让人过早的碰到挫折,丧失继续读下去的勇气和毅力。有时候,一本书是在太难了,大略读一遍只能把握其中一小部分信息,然而聊胜于无,把粗略阅读过程中的问题留到分析阅读过程中解决,也比从来没有浏览过一遍强。

\subsection{分析阅读}
\label{sec:orgheadline7}


分析阅读是解析一本书相当重要的一部分。在这个过程中我们更深入的回答前文提出的四个问题。可以这是个问题为导向展开分析阅读。在分析阅读的过程中,有一些原则,我认为非常的重要。把握这些原则后,我们回答所提的四个问题如虎添翼。
\subsubsection{透视一本书}
\label{sec:orgheadline4}


其实还是要回答前两个问题,作者谈了什么?是如何展开的?在这个透视一本书的过程中要完成的任务:

\begin{enumerate}
\item 知道书籍的性质,而且越早越好。这是一本理论的书籍还是一本技术的书籍,是一本哲学书籍还是一本科幻书籍。不同类型的书籍,其对同一个问题的解析也肯定不同,我们也会对其有不同的期待。比如大学里的《微积分》就侧重于数学技巧的传授,帮助学生掌握微积分这个工具,而《数学分析》则更偏向于分析,侧重于培养数学思维。所以工程类的学生大多修的是《微积分》,而数学专业的大多修的是《数学分析》。

\item 使用一段话叙述整本书的内容。其实负责任的作者在摘要和前言中会对整本书的内容进行高屋建瓴的概括。而读者一定要合上书本,用自己的话来概括,看看会不会出现偏颇和遗漏。很多人对于整体内容的认识不够深刻,往往会出现“我知道这本书在谈什么,但是说不上来”,这是一种自我欺骗,没有回答问题,根本没有对一本书进行整体的认识。

\item 将书中重要篇章列出来,说明他们是如何按顺序进行架构的。原因很简单,一本书肯定有多个部分构成,作者通过将这多个部分组合起来完成主题的论证。对于这个任务,可以按照一个固定的模式来完成,即:作者将全书分成四部分,每一部分的内容分别是什么;第一部分作者又分成几个分论点,每个论点是什么;作者先介绍第一部分为后一部分提供了什么帮助,先介绍第一个论点为后一个论点提供了什么帮助。

\item 找出作者要解决的问题。 作者写书之初脑海里就有一堆问题,而书中内容就是这些问题的答案。身为读者,有责任找出这些问题,并给出作者解决这些问题的过程。
\end{enumerate}

一本好书必然也是一本可读性很强的书,作者会尽全力帮助读者完成这些任务。以上四个任务帮助我们对一本书形成一个整体的认识。

\subsubsection{学会作者的语言}
\label{sec:orgheadline5}


透视一本书之后,我们需要深入每个论点内部,去探究我们感兴趣的每个细节,或者说探究支持作者主题的各个分论点。这个部分帮助我们回答:作者是如何展开核心主题的?涉及的子任务包括:

\begin{enumerate}
\item 找出关键字,并通过上下文用作者和我(读者)的共同语言来理解这个关键字。每个作者都有自己的喜好,在选择使用关键字方面肯定有浓重的个人色彩,读者只有学会了作者的语言,才能与之沟通。阅读一本比较难的书,书中必然有一些读者不能一读即懂的句子,也正是这些句子让书变难了。建立和作者的共同语言有助于加快对书中观点的理解。

\item 将书中重要的句子圈出来,找出其中的主旨。读完一本书,你的书上应该有很多标记才对,标记着作者的关键词,关键句子并附有你自己的理解。我发现有的人阅读一本书,自始至终都没有一个自己的印记。我不对此作任何评论,但是我不会这么做。我觉得一个读者对作者的最高敬意就是在他的书上勾勾画画并附上自己的理解和批评。这一步要求找出主旨,并将重要的句子圈出来方便了下一个任务。

\item 根据书中的主旨句,架构整本书的基本论述。阅读一本书不是从头到尾翻一遍,而是反复咀嚼的过程。书中的论点被圈出,然后单个的论点还要被反复分析,并考虑其中联系。把这些主旨句组织起来,架构文章的基本论述,是一个把书读薄的过程。

\item 确定哪些问题是作者已经解决的,哪些是作者没有解决的,哪些是作者也无法解决的。
\end{enumerate}
\subsubsection{评价一本书}
\label{sec:orgheadline6}


对一本书做出评价,回答了后两个问题:1)这本书有道理么?2)与我有什么关系?这本书能够对一本书做出评价,通常是作者比较喜欢见到的事情。但是读者在评价一本书的时候通常需要注意一些礼节。

\begin{enumerate}
\item 在你说出“我同意”,“我不同意”或者“我暂缓评论”之前,你一定要肯定的说“我了解了”。这点事非常重要的,我最反对在做报告的时候玩手机的人插嘴上来问一些在报告中已经被反复论及的问题。我会觉得没有被尊重。读书也是如此,许多人只是大略粗度一番便大放厥词,好像自己有多么博闻强识。不论同意和不同意都要说出理由来。

\item 当你不同意作者观点时,要理性的表达自己的见解,不要无理的辩驳或者争论。柏拉图在《会饮篇》中描述了阿加顿与苏格拉底的对话:

“我不能反驳你,苏格拉底”,阿加顿说,“让我们假设你说的对好了”。

“阿加顿,你该说你不能反驳真理,因为苏格拉底是很容易被反驳的”。
\end{enumerate}

柏拉图给了我们很多人忽视的一个忠告:大多数人以赢得争辩为目的,却没想到要学习的是真理。

\begin{enumerate}
\item 尊重知识与个人观点的不同,在做任何评断之前都要找出理论基础。事实上,由于解码角度的不同,我们对同一事物有不同的解读是完全有可能的。比如,对支付宝这个APP,我老婆会想到这是可以用来付款的软件;我会问这个APP是如何实现数据加密的,如何实现内部的诸多功能的。再比如,对一个儿童玩具,小孩子的解读就是会带来快乐的一个玩具;对于成年人的解读可能是这个玩具要花多少钱才能买到,现在的小孩真会玩;对于玩具工程师的解读可能是这个玩具是怎么做出来的。所以,就算观念不同,不必激动。
\end{enumerate}


\subsection{主题阅读}
\label{sec:orgheadline8}


我们把阅读书籍本身叫做内在阅读,使用辅助材料叫做外在阅读。就某一个主题收集多方面的材料,综合内在阅读和外在阅读的过程叫做主题阅读。在开始主题阅读之前,我们需要

\begin{enumerate}
\item 针对要研究的课题,找出一份文献目录,这些文献目录可以从老师那里得来,也可以从书籍后面的参考文献获得。

\item 浏览这份书目上的书,决定哪些与要研究的课题相关,并与要研究的课题建立清楚的联系。
\end{enumerate}

一旦开始主题阅读,就是一个旁征博引的过程,这也是做研究的关键部分了,大致的流程可以概括如下:

\begin{enumerate}
\item 对第一阶段确定的书籍进行检视阅读,定位于你的课题相关的章节。

\item 根据主题创造出一套中立的词汇,与作者达成共识--无论作者是否使用这些词汇,所有的作者或至少大部分作者都可以用这套词汇来解释。

\item 建立一个中立的主旨,列出一连串问题,无论作者是否明白的谈论过这些问题,他们都应该能为这些问题的解答提供资料

\item 界定主要和次要议题。然后将各个作者针对各个议题的见解主旨放在这些议题旁边。

\item 分析这些议题,对各个作者的观点进行讨论
\end{enumerate}


\section{从资料到知识}
\label{sec:orgheadline10}


到此,或简或繁的讨论了主动阅读的四个层次:1)基础阅读,2)检视阅读,3)分析阅读和4)主题阅读。本节讨论一些资料和知识的差别。下一节讨论阅读习惯的培养。这两节是比较独立的章节。

我们首先来讨论一下资料和知识的区别。或许你周围就有这样的朋友:他们具有很强的收集癖好---相关主题的电子书和影视材料几十G,用一个专门的移动硬盘装起来。这本无可厚非,问题是:时间久了,他们还真以为这些资料变成了内化的知识,若有人问他问题,他会丢过来一句话“你可以看看什么什么书?”。事实上,我问你是想听你对这个问题的见解而不是想知道那本书是怎么说的。没错,我曾经就是这样的人。有同学问我问题,一旦我答不上来,就说你去看看XXX。这是一种很不负责任的回答。

采铜在知乎上有一些列的文章《如何成为高段位的学习者》,里面详细的论述了资料和知识的不同。那么我的这篇博文到底资料是还是知识呢?对我来说是知识,对你来说是资料。因为这里是我综合了多个方面的资料,经过提炼加工用自己的语言表达出来的,期间包含对已有知识体系的调用和构建。于你来说可能就是一篇介绍如何读书的博文。

关于我是如何实现从资料到知识,我写了一篇博文《我的知识流》。这篇博文中介绍了我是如何“正儿八经”得处理我那些电子材料的。

\section{阅读习惯的培养}
\label{sec:orgheadline11}


习惯是对已有动作程序的反复操练。所谓艺术或者技巧,只属于那些能养成好习惯,而且能依照这些规则操作的人。这也是艺术家和普通工人的区别。

通过将动作内化为习惯,做事情的时候我们可以节省很多精力去回忆动作,因为这些已经属于我们身体的一部分。我们就有更多的意志力去发挥想象力,去深入思考,去干一些更“高级”的事情。

阅读和游泳一样,一个专家和一个初学者的做法是不一样的。在游泳时,初学者的动作极不协调,笨手笨脚,稍不留神还要喝口水漱漱口。学习游泳是一个很令人尴尬的事情,因为我们的身体已经习惯了在陆地上行走,要在水中借助水的浮力和波动水的推力前行,初学者的身体还没有准备好。但是游泳教练却可以在水中来去自如,简直“如鱼得水”。甚至有时候,一个游泳教练对一个新手也感觉难以理解,明明很简单的动作,为什么就学不会。这是因为,这些游泳动作已经内化为教练身体的一部分,当他促水的那一刻,他的身体就知道该如何协调不同于陆地上的动作,以保证身体不会下沉并顺利前行。而初学者则必须反复操练才可以。

我还要说说我使用Emacs的经历,Emacs是一款伟大的编辑器,基本上我每天打开电脑的第一时间开启的软件就是Emacs,关闭电脑前最后关闭的软件才是Emacs。然而刚刚还是学习Emacs时,我还是被Emacs怪异的快捷键和elisp的配置方式镇住了。然而,当我习惯了Emacs的快捷键和配置方式,我发现这是一款伟大的软件。那些操作流程完全被我的手指记忆,当我把手放到键盘上的时候,我根本不用想这个操作该按什么键。我的工作效率也被大大提高了。掌握了Emacs之后,我的诸多工作也都在这一个软件内完成,编写博客,发布博客,C代码,matlab代码,记笔记,写论文,甚至浏览网页,听音乐等等等等都在一个软件中完成,那种工作流的顺畅常常让我感觉到身心愉悦。

然而,要养成习惯除了不断操练,别无他法。一个牛人做事和一个新手做事的区别也就在于牛人根本不用去想规则,他已经内化了这些规则,而新手则要一步一步笨手笨脚的看着规则来。也就是说,知道规则和养成习惯是两码事。只有不停的操练规则,才能养成习惯。所以当你在选择规则的时候,切记选择那些高效的规则,选择那些具有长半衰期的规则。这也是为什么如果条件允许,大家都愿意找那些顶级的高手来学习,因为他内化的规则是高效的,学会之后具有较长的半衰期---在相当长时间内,我只要使用这些规则即可,而不用考虑去更新。在学习钢琴演奏的时候更是如此,一个人如果学习之初就掌握了较好的指法和手势,那么对于他以后的进步无疑有巨大的裨益。如果一开始的动作就不对,以后即使碰上大师,大师也是头疼的,甚至不愿教你。因为你不是一张白纸了,上面已经有了污点,弄干净是一件耗时耗力的事情。


再回到读书,会读书的好习惯也要及早养成,读书过程中那些优秀的规则也要不停的内化。这就是我写这篇博文的原因:通过写博客操练达到读书规则的内化。
\end{document}
