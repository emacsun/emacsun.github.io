% Intended LaTeX compiler: pdflatex
\documentclass[10pt,a4paper,UTF8]{article}
\usepackage{zclorg}
\author{zcl.space}
\date{}
\title{随机过程的定义}
\hypersetup{
 pdfauthor={zcl.space},
 pdftitle={随机过程的定义},
 pdfkeywords={},
 pdfsubject={},
 pdfcreator={Emacs 25.0.50.1 (Org mode 9.0.5)}, 
 pdflang={English}}
\begin{document}

\maketitle
按照事物发展变化的可预测与否,自然界中事物变化过程可以分两类。第一类具有确定的变化过程,具有必然的规律,可以用一个时间\(t\)的函数来描述。这类过程称为确定性过程。例如水平向前丢一个石头,其垂直方向的速度永远是\(v=gt\),\(g\)是重力加速度,\(t\)是时间。另一类过程则没有确定的变化形式,也就是说每次观测这个过程,其测量结果没有一个确定的变化规律,用数学语言说就是,这类事物的变化过程不能用一个时间\(t\)的确定函数来描述。如果对该事物的变化过程进行一次观测,可以得到\(t\)的一次函数,但是若对这个过程进行重复独立的观测,则每次得到的结果是不同的。从另一个角度来讲,如果我们固定某一个观察时刻\(t\),则事物在时刻\(t\)出现的状态是随机的。这类过程叫做随机过程。

虽然随机过程不能用一个确定的函数来描述,但是随机过程也是有规律的。学习随机过程的目标就是寻找如何描述一个随机过程,并研究这个随机过程的性质和规律。事实上,前人已经总结除了很多随机过程的模型供我们参考。

我们给出一些例子,描述随机过程。

\begin{instance}
首先给出的是伯努利过程。以掷硬币为例。设想每单位时间丢一次硬币,观察结果。如果出现正面记为 1,如果出现反面记为 0。一直丢下去,便可得到一个无穷序列\(\{x_{1},x_{2},\ldots \}\),则:
\begin{equation}
\label{eq:1}
\{x_{1},x_{2},\ldots \} = \{x_{n}:n=1,2,\ldots ;x_{n}=1 \quad or \quad x_{n} = 0\}
\end{equation}

因为每次抛掷的结果\(x_{n}\)是一个随机变量\(1\)或者\(0\),所以无穷次抛掷的结果是一随机变量的无穷序列。称随机变量的序列为随机序列,也可以说是随机过程。每次抛掷的结果与向后歌词抛掷的结果是统计独立的,并且\(x_{n}\)出现\(0\)或者\(1\)的概率与抛掷的时间\(n\)无关。设:
\begin{eqnarray}
\label{eq:2}
p\{x_{n} = 1\}&=& p \\
p\{x_{n} = 0\}&=&1- p
\end{eqnarray}
其中\(p\{x_{n} = 1\} = p\) 与\(n\)无关,且\(x_{i},x_{k},i\neq k\)是相互统计独立的随机变量。称具有这种特性的随机过程为伯努利型随机过程。

有许多实际问题可以归类到伯努利概率模型。如在数字通信中所传送的信号是脉冲信号,在某一时刻\(t\)可能出现脉冲也可能不出现脉冲,出现脉冲记为\(1\),不出现脉冲记为\(0\),则在\(t\)时刻信号的值\(x\)是一个随机变量,\(x_{t}\)有两个取值。如果在\(t_{1},t_{2},t_{3},\ldots\)时观察信号,则所得结果\(\{x_{1},x_{2},x_{3},\ldots \}\)。如果在\(t_{k}\)时刻出现\(1\)或者\(0\)的概率和观察的时刻\(t_{k}\)无关,在\(t_{i}\)时刻出现\(x_{i}\)和在\(t_{k}\)时刻出现\(x_{k}\)是相互独立的,并设\(P(x_{i} =1) = p,P(x_{i} = 0) = 1-p\)则\(p\)与\(i\)无关,且\(x_{i},x_{k}\)是相互独立的随机变量,这样形成的随机序列属于伯努利概型。
\end{instance}

\begin{instance}
正弦波过程。在振荡器的大批生产过程中抽出一台振荡器,它的输出波形为:
\begin{equation}
\label{eq:3}
x(t) = v\sin(\omega t + \phi)
\end{equation}

其中\(v\)是振幅,\(\omega\)是震荡角频率,\(\omega = 2\pi f\),\(f\)是频率,\(\phi\)为振荡器的初始相位。由于生产工艺的偏差,每个振荡器的振幅和频率与额定值都会有一定的偏差,各台的偏差是不一致的,也就是说\(v,\omega\)是随机变量,每一台的\(v,\omega\)是样本空间\((V,\Omega)\)的一个样本点。而且把振荡器接上电源,初始相位\(\phi\)也是随机的,所以每次对一台振荡器做实验,其输出电压的\(v,\omega,\phi\)为随机变量。不同振荡器在各次试验中其输出电压的时间函数虽然是正弦波,但因为\(v,\omega,\phi\)为随机变量,不同台不同次的输出可能均不相同,如果固定一个观察时刻,观察各台振荡器在这一时刻的电压,则\(x(t) = v\sin (\omega t + \phi)\)也是随机变量。在\(t\)时,\(x(t)\)的分布决定于\(t\)以及\(v,\omega,\phi\)的分布。

称\(x(t) = v\sin (\omega t + \phi)\)为正弦波过程,在这个过程中,\(t\)是一个参量,它可取\([0,\infty)\) 内的任意值。
\end{instance}

\begin{instance}
如果对晶体管的噪声进行测量,每隔单位时间去一个样本,则可在\(t=1,2,\ldots\)时刻测得一组无穷可列维随机矢量\(\{x_{1},x_{2},\ldots \}\)一次测量的结果为样本空间的一个点,每次测量的结果可能各不相同。我们每次测试的结果成为一个现实,或称为一个样本函数。另一方面,如果固定一个观测时刻,对噪声进行无穷次测量,则可以得到该时刻噪声的分布。如果固定第二个时刻,则测测该第二个时刻噪声的二维分布。如果固定\(n\)个时刻,则可测得\(n\)个时刻噪声的\(n\)维分布。
\end{instance}

根据上面的几个例子,我们可以对随机过程做一个概括的说明。
\begin{definition}
设\(\Omega, \mathcal{F},P\)是概率空间,\(T\)是直线上的参数集(可列的或不可列的),若对每一个\(t\in T\), \(\xi(\omega,t) = \xi_{t}(\omega)\)是随机变量,则称\(\{\xi(\omega,t),t\in T\}\)为该空间上的随机过程。
\end{definition}

随机过程是一个统称,根据\(T\)是否连续可以分为离散随机过程和连续随机过程。离散的随机过程也叫作随机序列或者时间序列。时间序列在金融分析中经常用到,我曾见过专门的金融时序分析方面的教材,但是没有深入研究,不知道是在那里是如何建模的。

我们把一次实验结果\(x_{k}(t),t\in T\)叫做随机过程的一个实现或者一个样本。通过本文我们可以发现可以通过两个角度去看随机过程,一个角度是对一个随机变量的无穷次测量,另一个角度是对无穷多个独立同分布的随机变量做一次测量。这两种方法各有所长,都统一与随机过程的定义中。
\end{document}
