\documentclass[10pt,a4paper,UTF8]{article}
\usepackage{zclorg}
\author{zcl.space}
\date{}
\title{度量空间以及由此引出的一些概念}
\hypersetup{
 pdfauthor={zcl.space},
 pdftitle={度量空间以及由此引出的一些概念},
 pdfkeywords={PMA},
 pdfsubject={本文是学习PMA第二章的一些总结},
 pdfcreator={Emacs 25.0.50.1 (Org mode 8.3.3)}, 
 pdflang={English}}
\begin{document}

\maketitle
\tableofcontents


\section{度量空间}
\label{sec:orgheadline1}


\textbf{定义} \emph{设\(X\)是一个集,它的元素叫做点,如果 \(X\)的任意两点 \(p\)和\(q\)联系于一个实数\(d(p,q)\),叫做从\(p\)到\(q\)的} \textbf{距离} , \emph{他们满足条件:}

\begin{enumerate}
\item 如果\(p\neq q\),那么 \(d(p,q)>0;d(p,p)=0\);
\item \(d(p,q)=d(p,p)\);
\item 对于任意 \(r\in X\), \(d(p,q)\leq d(p,r)+d(r,q)\);
\end{enumerate}

\emph{就称\(X\)是一个} \textbf{度量空间} 。

从定义可知,度量空间中有距离的概念。有距离才有度量,就像在Viterbi译码过程中,有欧氏距离或者汉明距离的定义,才会有幸存路径度量的概念。在度量空间中,距离函数的定义非常重要。最常见的度量空间是欧式空间\(R^{k}\),特别是 \(R^{1}\)(实数轴)和 \(R^{2}\)(复平面)。在\(R^{k}\)中,距离定义为 \[d(x,y) = |x-y|, x,y\in R^{k}\]

\section{由度量空间引出的一些定义}
\label{sec:orgheadline2}


度量空间中有一些非常重要的概念。这些概念互相关联为度量空间中的分析奠定了基础,特别是以下几个概念的引出更是环环相扣(以下提到的点和集都是度量空间 \(X\)中的点和集):

\begin{enumerate}
\item 点\(p\)的邻域 \(N_{r}(p)\)指的是满足条件 \(d(p,q) < r\)的一切点\(q\)所成的集。 \(r\)叫做\(N_{r}(p)\)的半径。
\item 点\(p\)叫做集\(E\)的极限点,如果 \(p\)的每个邻域都有一点 \(q\in E\) 而 \(q\neq p\)。
\item 如果 \(p\in E\)但 \(p\)不是 \(E\)的极限点,则\(p\)是\(E\)的孤立点。
\item \(E\)是闭集,如果 \(E\)的每个极限点都是\(E\)的点.
\item 点\(p\)叫做\(E\)的内点,如果存在\(p\)的一个邻域\(N\),有\(N\subset E\)。
\item \(E\)是开集,如果 \(E\)的所有点都是\(E\)的内点。
\item \(\{p|p\in X,\notin E\}\)构成\(E\)的余集\(E^{c}\).
\item \(E\)叫做完全的,如果\(E\)是闭集,并且\(E\)的每个点都是\(E\)的极限点。
\item \(E\)叫做有界的,如果有一个实数\(M\)和一个点\(q\in X\),使得一切\(p\in E\)都满足\(d(p,q) < M\)
\item \(E\)叫做在\(X\)中稠密,如果\(X\)的每个点或是\(E\)的极限点,或是\(E\)的点。
\end{enumerate}

显然,在\(R^{1}\)中,邻域就是开区间;在\(R^{2}\)中,邻域就是圆的内部(不包含圆的边)。这里特别要强调的极限点的定义,根据极限点的定义,极限点有可能不是\(E\)的点。比如,在\(R^{2}\)中,对于集合\(|z<1|\),满足\(z=1\)的那些点都是\(|z<1|\)的极限点,但是这些点却不是\(|z<1|\)的点。另外一个例子\(\{\frac{1}{n}|n=1,2,\ldots\}\)有一个极限点0,但是0却不是这个集合中的点。所以,对于集合拥有极限点和集合包含极限点是两个概念。

\section{与开集闭集有关的定理及其证明}
\label{sec:orgheadline3}


在Rudin的教材中,从度量空间以及刚才引出的那些定义出发,还有几个定理,如下:

\textbf{定理} 邻域必是开集


\textbf{证明} :假设有一邻域 \(E= N_{r}(p)\),令 \(q\)是 \(E\)中的任意一点,于是有一正实数 \(h\)使得 \[d(p,q) = r-h\] 对于一切 \(d(q,s) < h\) 的点 \(s\),我们有 \[ d(p,s) \leq d(p,q) + d(q,s) < r-h + h = r \] 所以\(s\in E\)。 因此, \(q\)是\(E\)的内点。

这个证明相当简洁,在证明过程中,两次用到邻域概念,一次用到开集概念。巩固了对开集和邻域定义的理解。

\textbf{定理} 如果\(p\)是集\(E\)的一个极限点,那么\(p\)的每个邻域都含有\(E\)的无限多个点。

\textbf{证明} :假设 \(p\)的某个邻域\(N\)只含有 \(E\) 的有限多个极限点, 令 \(q_{1},\ldots, q_{n}\)是 \(N\cap E\) 中这有限个异于 \(p\) 的点。 又令 \[r = \min_{1\leq m \leq n} d(p,q_{m})\] 显然有 \(r>0\)。那么邻域 \(N_{r}(p)\)中不能再含有\(E\)的点\(q\)并且\(q\neq p\)。所以 \(p\)不是 \(E\)的极限点,矛盾。

\textbf{定理} 有限的点集没有极限点

\textbf{定理} \(E\)是开集,当且仅当它的余集是闭集。

\textbf{证明} :首先假设 \(E^{c}\)是闭集,我们证明\(E\)是开集。假设\(x\in E\),所以\(x\notin E^{c}\)。另外\(x\)也不可能是\(E^{c}\)的极限点(因为我们假设\(E^{c}\)是闭集,闭集所有的极限点都是该集合的点。)。于是\(x\)有一个邻域\(N\),使得\(N\cap E=\varnothing\),即:\(N\subset E\)。所以\(x\)是\(E\)的内点,所以\(E\)是开集。

反过来,假设\(E\)是开集,我们证明\(E^{c}\)是闭集。假设\(x\)是\(E^{c}\)的一个极限点,那么\(x\)的每个邻域都含有\(E^{c}\)的点,所以\(x\)不是\(E\)的内点。因为\(E\)是开集,所以\(x\in E^{c}\),\(E^{c}\)是闭集。

\textbf{这个定理的证明过程紧扣开集,闭集,极限点和内点的定义}

\textbf{定理} 设 \(\{E_{\alpha}\}\)是若干(有限个或无限多个)集\(E_{\alpha}\)的一个组,那么\[(\underset{\alpha}{\cup}E_{\alpha})^{c} = \underset{\alpha}{\cap}(E_{\alpha}^{c})\]

\textbf{证明} :令 \(A = (\underset{\alpha}{\cup}E_{\alpha})^{c}\), \(B =  \underset{\alpha}{\cap}(E_{\alpha}^{c})\),若\(x\in A\),则\(x\notin (\underset{\alpha}{\cup}E_{\alpha})\),对于任意的\(\alpha\),有\(x\notin E_{\alpha}\),从而\(\forall \alpha, x\in E_{\alpha}^{c}\),所以\(x\in E_{\alpha}^{c}\),即 \(A\subset B\).

反过来,如果\(x\in B\),那么对于每个\(\alpha\),\(x\in E_{\alpha}^{c}\)。也即对于每个\(\alpha\),\(x\notin E\)。因此 \(x\notin \underset{\alpha}{\cup} E_{\alpha}\)。即\(x\in (\underset{\alpha}{\cup}E_{\alpha})^{c}\),于是\(B\subset A\)

\textbf{定理} (a) 任意一组开集\(\{G_{\alpha}\}\)的并 \(\underset{\alpha}{\cup} G_{\alpha}\)是开集。 (b) 任意一组闭集\(\{F_{\alpha}\}\)的交 \(\underset{\alpha}{\cap} F_{\alpha}\)是闭集。(c) 任意一组有限个开集 \(G_{1},\ldots , G_{n}\)的交 \(\bigcup\limits_{i=1}^{n} G_{i}\)是开集。(d) 任意一组有限个闭集 \(F_{1},\ldots, F_{n}\)的并 \(\bigcap\limits_{i=1}^{n}F_{i}\)是闭集。

\textbf{证明} :令\(G=\underset{\alpha}{\cup} G_{\alpha}\)。如果\(x\in G\),就有某个\(\alpha\),使得 \(x\in G_{\alpha}\)。从开集的定义出发(如果集合中所有的点都是内点,则该集合为开集),我们知道 \(x\)是\(G_{\alpha}\)的内点,从而\(x\)也是\(G\)的内点。由于\(x\)的任意性,所以\(G\)是开集。 这样我们就证明了定理的(a)。

接下来我们证明(b): \(F_{a}\)是闭集,闭集的余集是开集,意味着 \(F_{a}^{c}\)是开集,根据(a),我们有 \(\underset{\alpha}{\cup}F_{\alpha}^{c}\)是开集。我们还知道 \textbf{设 \(\{E_{\alpha}\}\)是若干(有限个或无限多个)集\(E_{\alpha}\)的一个组,那么\((\underset{\alpha}{\cup}E_{\alpha})^{c} = \underset{\alpha}{\cap}(E_{\alpha}^{c})\)} 。 所以 \((\underset{\alpha}{\cap})^{c}\)是开集。开集的余集是闭集,意味着 \(\underset{\alpha}{\cap}F_{\alpha}\)是闭集。

\section{闭包及相关定理}
\label{sec:orgheadline4}


今天还不是很累,觉得应该可以把闭包和相关的定理给学习完毕。

\textbf{闭包} 设\(X\)是度量空间,如果 \(E\subset X\), \(E^{'}\)表示 \(E\)在\(X\)中所有极限点组成的集,那么,把\(\overline{E} = E\cup E^{'}\)叫做 \(E\)的闭包。

解读:\(E^{'}\)中的点不一定属于\(E\)。\(E^{'}\)中的点是\(E\)在\(X\)中的极限点,根据极限点的定义,我们知道\(E\)的极限点不一定属于\(E\)。从而,\(E^{'}\)中的点也不一定属于\(E\)。\(E\)的闭包\(\bar{E}\)包含的点有属于\(E\)的点,也有不属于E的点。\(\bar{E}\)中属于\(E\)的那些点,或者是\(E\)的极限点,或者是\(E\)的孤立点。总之\(E\subset \bar{E}\),但是\(\bar{E}\)不一定等于\(E\)。

\textbf{定理} 设\(X\)是度量空间,而\(E\subset X\),那么 (a) \(\bar{E}闭\) ;(b) \(E=\bar{E}\)当前仅当\(E\)闭;(c) 如果闭集\(F\subset X\)且\(E\subset F\),那么\(\bar{E}\subset F\)。 由(a)和(c),\(\bar{E}\)是\(X\)中包含\(E\)的最小闭子集。

\textbf{证明} 如果\(p\in X\)而\(p\notin \bar{E}\),根据闭包的定义,\(p\)既不是\(E\)的点,也不是\(E\)的极限点。因此,\(p\)有某个邻域与\(E\)不交。所以\(\bar{E}\)的余集是开集,因此\(\bar{E}\)是闭集。

证明(b):如果\(E=\bar{E}\),(a)表明\(E\)闭。如果\(E\)闭,那么\(E\)的极限点是\(E\)的点,那么\(E^{'}\subset E\)。所以\(\bar{E} = E\)。

证明(c):如果\(F\)闭,且\(F\subset X\),则\(F^{'}\subset F\)。根据闭集定义,\(F\)的所有极限点都属于\(F\),即\(F^{'}\subset F\),因此\(E^{'}\subset F\),于是\(\bar{E}\subset F\)。

由(a)和(c),我们知道 \(\bar{E}\)是\(X\)中包含\(E\)的最小闭子集。具体为,根据(c),由于\(F\)的任意性,\(\bar{E}\)属于\(F\)。根据定义\(\bar{E} = E\cup E^{'}\),我们知道\(E\subset \bar{E}\)。可以说,除了闭包\(\bar{E}\)外,任何包含\(E\)的闭集,都至少包含了闭包\(\bar{E}\)。这样说或许有些不严密,但是是一个理解过程。

\textbf{解读} 上面这个定理告诉我们,闭包是包含该集合的最小闭集。上述定理也提出了一种从一个集合构建包含该集合最小闭集的步骤。

\textbf{定理} 设\(E\)是一个非空实数集,上有界,令\(y=\sup E\),那么\(y\in \bar{E}\)。特别的,如果\(E\)闭,那么,\(y\in E\)

\textbf{证明} 如果\(y\in E\),那么\(y\in \bar{E}\)。接下来我们考虑\(y\notin E\)的情形,对于每个\(h>0\),存在\(x\in E\),使得\(y-h < x < y\)。因为如果这样的\(h\)不存在的话,\(y-h\)就是\(E\)的上界了。所以\(y\)是\(E\)的极限点。

\textbf{解读} 这是根据极限点的定义来的。(点\(p\)叫做集\(E\)的极限点,如果点\(p\)的每个邻域都含有一点\(q\in E\)而\(q\neq p\))。鉴于\(h\)的任意性和\(x\)的任意性,在\(y\)的任意邻域内,总有一点属于\(E\),所以\(y\)符合极限点的定义,故\(y\)是\(E\)的极限点。

定义\(X\)是度量空间,设\(E\subset Y\subset X\),我们说\(E\)是\(X\)的开子集,就是说给每个\(p\in E\)配备一个正数\(r\),使得\(d(p,q) < r\) 和 \(q\in X\) 能保证\(q\in E\)。\(p\)是\(E\)的内点。特别的,如果能给每个\(p\in E\)配备一个\(r > 0\), 当\(d(p,q) < r\)且\(q\in Y\)时,就有\(q\in E\),我们就说\(E\)关于\(Y\)是开的。一个集合可以关于\(Y\)是开的,然而却不是\(X\)的开子集。

比如开区间\((a,b)\subset R^{1} \subset {R^{2}}\)。显然,对于每个\(p\in (a,b)\),配备一个正数\(r\),使得\(d(p,q) < r\)和\(q\in R^{2}\),但是我们不能保证\(q\in E\)。因此\((a,b)\)不是\(R^{2}\)的开子集。但是对于每个\(p\in (a,b)\),配备一个整数\(r\),使得\(d(p,q) < r\)和\(q\in Y\)。即\((a,b)\)关于\(Y\)是开的。
\end{document}
