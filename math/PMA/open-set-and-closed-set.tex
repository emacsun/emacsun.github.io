% Intended LaTeX compiler: pdflatex
\documentclass[10pt,a4paper,UTF8]{article}
\usepackage[utf8]{inputenc}
\usepackage[T1]{fontenc}
\usepackage{graphicx}
\usepackage{grffile}
\usepackage{longtable}
\usepackage{wrapfig}
\usepackage{rotating}
\usepackage[normalem]{ulem}
\usepackage{amsmath}
\usepackage{textcomp}
\usepackage{amssymb}
\usepackage{capt-of}
\usepackage{hyperref}
\author{zcl.space}
\date{}
\title{Open sets and closed sets}
\hypersetup{
 pdfauthor={zcl.space},
 pdftitle={Open sets and closed sets},
 pdfkeywords={PMA},
 pdfsubject={},
 pdfcreator={Emacs 25.0.50.1 (Org mode 9.1.2)},
 pdflang={English}}
\begin{document}

\maketitle
\tableofcontents
In all the following definitions the term \textbf{set} means \textbf{subset of} \(\mathbb{R}^m\).

\begin{definition}
A set  \(U\) is \textbf{open} iff for every \(p\in U\)
there exists a \(\delta > 0\) such that \(B(p,\delta)\subseteq U\).
\end{definition}


\begin{tikztheorem}
The collection of all open sets in \(\mathbb{R}^m\) satisfies the following conditions:
\begin{enumerate}
\item The  set  \(\mathbb{R}^m\) and the empty set \(\emptyset\) are both open.
\item The intersection of a finite collection of open sets is open.
\item The union of an arbitrary collection of open sets is open.
\end{enumerate}
\end{tikztheorem}
\begin{proof}
 The set \(\mathbb{R}^m\) is open because \(B(p,1)\subseteq\mathbb{R}^m\) for \(p\in\mathbb{R}^m\).
The empty set is open because for every \(p\in\emptyset\)
satisfies the required condition -- or any other condition --  since
\textbf{`false \(implies\) anything' is true}.
To prove\textasciitilde{}(2) assume \(U\) is open.  Then for  every point \(p\in U\) there is a \(\delta=\delta_p\)
such that \(B(p,\delta_p)\subseteq U\).\footnote{This is actually an example of an application of the Axiom of Choice.}

It follows that
$$
       U=\bigcup_{p\in U} B(p,\delta_p),
$$
i.e. that \(U\) is an union of balls. A union of unions is a union:
$$
     \bigcup_{i\in I}\bigcup_{j\in I_j} B_{ij}=\bigcup_{(i,j)\in K} B_{ij},
     \qquad
     K:=\{(i,j): i\in I,\; j\in I_j\}
$$
so\textasciitilde{}(2) follows.

To prove\textasciitilde{}(3) assume that \(U_1,U_2,\ldots U_m\) are open and choose an arbitrary point \(p\in\bigcap_{i=1}^m U_i\). Then \(p\in U_i\) so there is a \(\delta_i > 0\) with \(B(p,\delta_i)\subseteq U_i\).

Let \(\delta=\min(\delta_1,\ldots,\delta_m)\). Then
$$
   B(p,\delta)\subseteq \bigcap_{i=1}^m B(p,\delta_i)\subseteq \bigcap_{i=1}^m U_i
$$
as required.
\end{proof}
\begin{definition}
A set \(W\subseteq X\) is called \textbf{relatively open} in \(X\) iff for every \(p\in W\) there exists a \(\delta > 0\) such that \(B_X(p,\delta)\subseteq W\).
\end{definition}

A set \(U\subseteq\mathbb{R}^m\) is open if and only if it is relatively open in \(\mathbb{R}^m\).
For this reason many theorems can be generalized by systematically replacing \(\mathbb{R}^m\) by \(X\), \(B(p,\delta)\) by \(B_X(p,\delta)\), and the word \{\em open\} by the phrase \{\em relatively open in \(X\)\}.

\begin{tikztheorem}
A set \(W\) is relatively open in \(X\) if and only if \(W=X\cap U\) for some  open set \(U\subset \mathbb{R}^n\).
\end{tikztheorem}

\begin{tikztheorem}
A map \$f:X\(\to\) \$Y is continuous if and only if the inverse image \(f^{-1}(V)\) of every relatively open subset \(V\) of \(Y\) is a relatively open subset of \(X\).
\end{tikztheorem}

\begin{tikztheorem}
Assume that \(f:X\to Y\) and \(g:Y\to Z\) are continuous. Then \(g\circ f:X\to Z\) is continuous.
\end{tikztheorem}

\begin{definition}
A set \(X\) is \jdef{closed} iff its complement \(\mathbb{R}^n\setminus X\) is open.
\end{definition}

\begin{tikztheorem}
The collection of all closed sets in \(\mathbb{R}^n\) satisfies the following conditions:
\begin{enumerate}
\item The  set  \(\mathbb{R}^n\) and the empty set \(\emptyset\) of both closed.
\item The intersection of an arbitrary collection of closed sets is closed.
\item The union of a finite collection of closed sets is closed.
\end{enumerate}
\end{tikztheorem}

\begin{tikztheorem}
A set \(S\) is closed if and only if it is closed under limits of sequences, i.e. whenever \(\lim_{n\to\infty} p_n=p\)  and each \(p_n\in S\) we have \(p\in S\).
\end{tikztheorem}
\begin{proof}
To prove \{\em only if\} assume that \(S\) is closed, that \(\lim_{n\to\infty} p_n=p\),  and that each \(p_n\in S\). If \(p\notin S\) then \(p\in\mathbb{R}^m\setminus S\).

As this set is open  there is a \(\delta > 0\) such that \(B(p,\delta)\subset \mathbb{R}^m\setminus S\).
As the sequence converges to \(p\) there is an \(N\) such that \(p_n\in B(p,\delta)\) for \(n > N\) contradicting the
hypothesis that \(p_n\in S\). To prove \{\em if\} assume that \(S\) is not closed.
Then \(\mathbb{R}^m\setminus S\) is not open so there is a point \(p\in\mathbb{R}^m\setminus S\) such that
\(B(p,\delta)\not\subseteq \mathbb{R}^m\setminus S\) for every \(\delta > 0\). In particular for \(\delta=1/n\) there is a point
\(p_n\in B(p,1/n)\) (i.e. \(|p_n-p|<1/n\)) such that \(p_n\notin \mathbb{R}^m\setminus S\), i.e. \(p_n\in S\). Thus
\(\lim_{n\to\infty}p_n=p\) and \(p\notin S\) as desired.
\end{proof}

Let  \(S\subseteq\mathbb{R}^n\). For any point \(p\in\mathbb{R}^n\) exactly one of the following alternatives holds:

\begin{enumerate}
\item \(B(p,\delta)\subseteq S\) for some \(\delta > 0\).
\item \(B(p,\delta)\subseteq \mathbb{R}^n\setminus S\) for some \(\delta > 0\).
\item \(B(p,\delta)\cap S\ne\emptyset\) and \(B(p,\delta)\cap (\mathbb{R}^n\setminus S)\ne\emptyset\) for all \(\delta > 0\).
\end{enumerate}

The \textbf{interior} of  \(S\) is the set \(\mathrm{int}(S)\) of all points \(p\)  where\textasciitilde{}(1) holds,
the \textbf{exterior}  of \(S\) is the set  \(\mathrm{ext}(S)\) of all points \(p\)  where\textasciitilde{}(2) holds, and
the \textbf{boundary} of a set \(S\) is the set \(\mathrm{bdry}(S)\) of all points \(p\)  where\textasciitilde{}() holds.
The ambient space \(\mathbb{R}^n\) may be written as the pairwise disjoint union \[ \mathbb{R}^n=\mathrm{int}(S)\cup\mathrm{ext}(S)\cup\mathrm{bdry}(S).\]


The notations \[ \stackrel{\circ}{S}\;:=\mathrm{int}(S),\qquad \partial S := \mathrm{bdry}(S)\]
are commonly used.


For the half open interval \(S=[a,b)\subseteq\mathbb{R}\) we have \[\mathrm{int}(S)=(a,b),\qquad \mathrm{ext}(S)=(-\infty,a)\cup (b,\infty),\qquad \mathrm{bdry}(S)=\{a,b\}.\]

\begin{definition}
A set \(S\subseteq\mathbb{R}^n\) is \jdef{closed} iff its complement
\(\mathbb{R}^n\setminus S\) is open. The \jdef{closure} of the set \(S\)
is the set
$$
   \mathrm{cl}(S):=\bar{S}:=S\cup\mathrm{bdry}(S).
$$
\end{definition}

\begin{tikztheorem}
The interior \(\mathrm{int}(S)\) of \(S\) is the largest open set contained in \(S\)  and closure  \(\bar{S}\) of \(S\)  is the smallest closed set containing \(S\).
\end{tikztheorem}
\end{document}
