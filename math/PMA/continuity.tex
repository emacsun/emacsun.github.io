% Intended LaTeX compiler: pdflatex
\documentclass[10pt,a4paper,UTF8]{article}
\usepackage[utf8]{inputenc}
\usepackage[T1]{fontenc}
\usepackage{graphicx}
\usepackage{grffile}
\usepackage{longtable}
\usepackage{wrapfig}
\usepackage{rotating}
\usepackage[normalem]{ulem}
\usepackage{amsmath}
\usepackage{textcomp}
\usepackage{amssymb}
\usepackage{capt-of}
\usepackage{hyperref}
\author{zcl.space}
\date{}
\title{Continuity}
\hypersetup{
 pdfauthor={zcl.space},
 pdftitle={Continuity},
 pdfkeywords={PMA},
 pdfsubject={},
 pdfcreator={Emacs 25.0.50.1 (Org mode 9.1.2)},
 pdflang={English}}
\begin{document}

\maketitle
\tableofcontents
Throughout this chapter
\(f:X\to Y\) where \(X\subseteq\mathbb{R}^n\) and \(Y\subseteq\mathbb{R}^m\).
\begin{definition}
The map  \(f\) is said to be \jdef{continuous} at
a point  \(p\in X\)  iff
for every \(\epsilon > 0\) there exists \(\delta > 0\)
such that \(f(B_X(p,\delta))\subseteq B(f(p),\epsilon)\).
\end{definition}
\begin{tikztheorem}
The map \(f\) is continuous at \(p\in X\) if and only if
for every sequence \(\{p_n\}\) of points in \(X\) we have
\begin{equation}\label{eq:limpif}
    \lim_{n\to\infty}p_n=p\implies \lim_{n\to\infty}f(p_n)=f(p).
\end{equation}
\end{tikztheorem}

\begin{proof}
Assume \(f\) is continuous at \(p\).
Choose sequence \(\{p_n\}\) of points in \(X\).
Assume
\begin{equation}\label{eq:limp}
\lim_{n\to\infty}p_n=p.
\end{equation}
Choose \(\epsilon > 0\). Because \(f\) is assumed to be
continuous at \(p\) there is a \(\delta > 0\) such that
for all \(q\in X\)
\%
\begin{equation}\label{eq:imp}
|q-p| < \delta\implies |f(q)-f(p)| < \epsilon.
\end{equation}
\%
there is an \(N\) such that
\(|p_n-p| < \delta\) for \(n > N\). Hence
\(|f(p_n)-f(q)| < \epsilon\) for \(n > N\). This proves
\%\%
\begin{equation}\label{eq:limf}
\lim_{n\to\infty} f(p_n)=f(p).
\end{equation}
as required.

 Assume that \(f\) is not continuous at \(p\in X\).
Then there is an \(\epsilon > 0\) such that for every \(\delta > 0\) there
is a \(q\in X\) such that \[  |q-p| < \delta\mbox{ but }  |f(q)-f(p)|\ge\epsilon\] .

In particular, for each \(n\in\mathbb{Z}^+\) there is a \(q_n\) such that \[  |q_n-p| < \frac1n \mbox{ but }  |f(q_n)-f(p)|\ge\epsilon \].
\end{proof}
\begin{definition}
The map  \(f\) is said to be \jdef{continuous}
iff it is continuous at every  point  of \(X\), i.e. iff
$$
\forall p\in X\;\forall \epsilon > 0\;\exists\delta > 0 \mbox{ such that }
f(B(p,\delta))\subseteq B(f(p),\epsilon).
$$
The map  \(f\) is said to be \textbf{uniformly continuous}
iff
$$
\forall \epsilon > 0\;\exists\delta > 0\; \mbox{ such that } \forall p\in X
\mbox{ we have } f(B(p,\delta))\subseteq B(f(p),\epsilon).
$$
(For continuity \(\delta=\delta(p,\epsilon)\);
 for uniform continuity \(\delta=\delta(\epsilon)\).)
\end{definition}
\begin{proof}
\note{(Buck Theorem~23 and its corollary on pages~62-63.)}
\end{proof}

\begin{proposition}
 If \(f:X\to Y\) and \(g:Y\to Z\) are both continuous, then so is the
composition \(g\circ f:X\to Z\).
\end{proposition}

\begin{proof}
 Choose \(p_0\in X\) and \(\epsilon > 0\). As \(g\) is continuous there exists \(\eta > 0\)
such that \(|g(q)-g(f(p_0))| < \epsilon\) whenever \(q\in Y\) and \(|q-f(p_0)| < \eta\).
As \(f\) is continuous, there exists \(\delta > 0\) such that \(|f(p)-f(p_0)| < \eta\)
whenever \(p\in X\) and \(|p-p_0| < \delta\). For \(p\in X\) we have \(q=f(p)\in Y\) so

\[ |p-p_0| < \delta\implies |f(p)=f(p_0)| < \eta\implies |(g\circ f)(p)-(g\circ f)(p_0)| < \epsilon\]


as required.
\end{proof}



  The map \(f\) is said to be \textbf{Lipschitz} iff there is
a constant \(M\) such that \[   |f(p)-f(q)|\le M|p-q|\]
for all \(p,q\in X\). A Lipschitz function is uniformly continuous.
(Proof: \(\delta=\epsilon/M\).)
\end{document}
