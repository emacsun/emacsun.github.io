\documentclass[10pt,a4paper,UTF8]{article}
\usepackage{zclorg}
\author{emacsun}
\date{}
\title{《具体数学》读后感}
\hypersetup{
 pdfauthor={emacsun},
 pdftitle={《具体数学》读后感},
 pdfkeywords={},
 pdfsubject={},
 pdfcreator={Emacs 25.0.50.1 (Org mode 8.3.2)}, 
 pdflang={English}}
\begin{document}

\maketitle\xiaosihao
\tableofcontents\newpage\newpage


\section{引言}
\label{sec:orgheadline1}


数学一直是通信工程师的短板(至少对我来说)。倘若立志在通信理论有所作为,数学素养恐怕要达到数学专业研究生的水平,甚至涉猎的方面要更广。因为通信不仅和数学紧密联系,和计算机理论亦是如此。最新的通信理论发展已经模糊了数学,计算机和信息理论的界限。

我对数学的学习,从来不曾终止过。然而由于学习方法不系统加上贪求过多,导致收效甚微。在阅读了不少关于学习方法以及人类思维意识方面的书籍之后,我决定逐点突破不再贪求过多。我相信当我积累到一定量之后,我所掌握的知识不再是一盘散沙,而会阡陌交通彼此互联。当然,投入大量时间是必须的。

目前而言,我的数学计划包括:1)重温微积分,2)学习高德纳的《具体数学》,作为计算机算法分析的数学基础。我还想把《抽象代数》和《泛函分析》重新复习一下。
\end{document}
