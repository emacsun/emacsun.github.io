% Intended LaTeX compiler: pdflatex
\documentclass[10pt,a4paper,UTF8]{article}
\usepackage{zclorg}
\date{}
\title{本征空间}
\hypersetup{
 pdfauthor={},
 pdftitle={本征空间},
 pdfkeywords={},
 pdfsubject={},
 pdfcreator={Emacs 25.0.50.1 (Org mode 9.0.5)},
 pdflang={English}}
\begin{document}

\maketitle
\begin{definition}
设\(T\in \mathcal{L}(V)\)且\(\lambda \in \mathbf{F}\),\(T\)的相应于\(\lambda\)的本征空间定义为:
\begin{equation}
\label{eq:1}
E(\lambda,T) = \mathrm{null}(T-\lambda I)
\end{equation}
也就是说,\(E(\lambda,T)\)是\(T\)的相应于\(\lambda\)的全体本证向量加上\(0\)构成的集合。
\end{definition}

对于\(T\in \mathcal{L}(V)\)和\(\lambda \in \mathbf{F}\),本征空间\(E(\lambda,T)\)是\(V\)上的子空间,因为线性映射的零空间都是\(V\)的子空间。由定义可知,\(\lambda\)是\(T\)的特征值当且仅当\(E(\lambda,T)\neq \{0\}\)。

\begin{theorem}
设\(V\)是有限维的,\(T\in \mathcal{L}(V)\),设\(\lambda_{1},\ldots ,\lambda_{m}\)是\(T\)的互异的本征值,则:
\begin{equation}
\label{eq:2}
E(\lambda_{1},T) + \ldots E(\lambda_{m},T)
\end{equation}
是直和,此外:
\begin{equation}
\label{eq:3}
\dim E(\lambda_{1},T) + \ldots + \dim E(\lambda_{m},T) \leq =dim V
\end{equation}
\end{theorem}

\begin{proof}
假设
\begin{equation}
\label{eq:4}
u_{1} +\ldots + u_{m} = 0
\end{equation}
其中每个\(u_{j}\)包含于\(E(\lambda_{j},T)\),因为相应与互异的特征值的特征向量是线性无关的,所以上式中\(u_{j} = 0,\forall j\)。因此\(E(\lambda_{1},T) + \ldots + E(\lambda_{m},T)\)是直和。

现在有:
\begin{equation}
\label{eq:5}
\dim E(\lambda_{1},T) + \ldots + \dim E(\lambda_{m},T) = \dim (E(\lambda_{1},T) \oplus + \ldots + \oplus E(\lambda_{m},T)) \leq \dim V
\end{equation}
\end{proof}
算子\(T\in \mathcal{L}(V)\)称为可对角化的,如果概算自关于\(V\)的某个基有对角矩阵。

\begin{instance}
定义\(T\in \mathcal{L}(\mathbf{R}^{2})\)为\(T(x,y)=(41x+7y,-20x+74y)\).\(T\)关于\(\mathbf{R}^{2}\)的标准基的矩阵为:
\begin{equation}
\label{eq:6}
\begin{bmatrix}
41 & 7 \\
-20 & 74
\end{bmatrix}
\end{equation}
这不是一个对角矩阵,但是\(T\)可以对角化,其关于\((1,4),(7,5)\)的矩阵为:
\begin{equation}
\label{eq:7}
\begin{bmatrix}
69 & 0 \\
0 & 46
\end{bmatrix}
\end{equation}
\end{instance}

\begin{theorem}
设\(V\)是有限维的,\(T\in \mathcal{L}(V)\),用\(\lambda_{1},\ldots ,\lambda_{m}\)表示\(T\)的所有互异的本征值。则下列条件等价:
\begin{enumerate}
\item \(T\)可对角化;
\item \(V\)有由\(T\)的本证向量构成的基;
\item \(V\)有在\(T\)下不变的一维子空间\(U_{1},\ldots ,U_{n}\)使得\(V = U_{1}\oplus \ldots \oplus U_{n}\)
\item \(V=E(\lambda_{1},T) \oplus \ldots + E(\lambda_{m},T)\)
\item \(\dim V = \dim E(\lambda_{1},T) + \ldots + \dim E(\lambda_{m},T)\)
\end{enumerate}
\end{theorem}

\begin{proof}
算子\(T\in \mathcal{L}(V)\)关于\(V\)的基\(v_{1},\ldots ,v_{n}\)有对角矩阵:
\begin{equation}
\label{eq:8}
\begin{bmatrix}
\lambda_{1} & & 0 \\
&\ddots & \\
0&&\lambda_{n}
\end{bmatrix}
\end{equation}
显然有\(Tv_{j} = \lambda_{j}v_{j}\)。即,这些基也是\(T\)的本证向量。所以\(V\)的这些基由\(T\)的本证向量构成。

假设第二步成立,则\(V\)有一个\(T\)的本证向量构成的基\(v_{1},\ldots ,v_{n}\),对每个\(j\),设\(U_{j} = \mathrm{span}(v_{j})\),显然每个\(U_{j}\)都是\(V\)的一维子空间且在\(T\)下不变。因为\(v_{1},\ldots ,v_{n}\)是\(V\)的基,所以\(V\)中每个向量都可以唯一的写成\(v_{1},\ldots ,v_{n}\)的线性组合。也就是说\(V\)中的每个向量都可以写成\(u_{1}+\ldots +u_{n}\)的线性组合,其中每个\(u_{j}\in U_{j}\),于是\(V= U_{1}\oplus \ldots \oplus U_{m}\)。

假设第三步成立,则\(V\)有在\(T\)下不变的一维子空间\(U_{1},\ldots ,U_{n}\)使得\(V=U_{1}\oplus + \ldots + \oplus U_{n}\)。假设\(\forall~j,v_{j}\in U_{j},u_{j}\neq 0\),则每个\(v_{j}\)都是\(T\)的特征向量。因为\(V\)中的每个向量都可以唯一地写成\(u_{1}+ \ldots u_{n}\)的形式,所以\(v_{1},\ldots ,v_{n}\)是\(V\)的基。

现在我们证明了第一步,第二步和第三部是等价的,

现在证明第二步蕴含第四步,第四步蕴含第五步,第五步蕴含第二步。

假设第二步成立,则\(V\)有一个由\(T\)的本证向量组成的基。于是,\(V\)中每个向量都是\(T\)的本证向量的线性组合,因此:
\begin{equation}
\label{eq:9}
V= E(\lambda_{1},T) + \ldots + E(\lambda_{n},T)
\end{equation}
又因为\(\lambda_{1},\ldots ,\lambda_{n}\)是互异的特征值,所以:
\begin{equation}
\label{eq:10}
V= E(\lambda_{1},T) \oplus \ldots \oplus E(\lambda_{n},T)
\end{equation}
第四步成立,则根据2.C.16,第五步成立。
\end{proof}

\begin{instance}
证明由\(T(w,z) = (z,0)\)定义的算子\(T\in \mathcal{L}(\mathbf{C}^{2})\)不可对角化
\end{instance}

\begin{proof}
容易验证\(0\)是\(T\)的唯一本征值且\(E(0,T) = \{(w,0)\in \mathbf{C}^{2}:w\in \mathbf{C}\}\),根据以上的证明,\(T\)不可对角化。
\end{proof}

\begin{theorem}
若\(T\in \mathcal{L}(V)\)有\(\dim V\)个互异的本征值,则\(T\)可对角化。
\end{theorem}

\begin{proof}
设\(T\in \mathcal{L}(V)\)有\(\dim V\)个互异的本征值\(\lambda_{1},\ldots ,\lambda_{\dim V}\)对每个\(j\),设\(v_{j}\in V\)是相应于本征值\(\lambda_{j}\)的本证向量。因为相应与互异的特征值的特征向量是线性无关的,所以\(v_{1},\ldots ,v_{\dim V}\)线性无关。\(V\)中\(\dim V\)个向量组成的线性无关组是\(V\)的基。
\end{proof}
\end{document}
