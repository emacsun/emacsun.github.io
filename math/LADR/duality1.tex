% Intended LaTeX compiler: pdflatex
\documentclass[10pt,a4paper,UTF8]{article}
\usepackage{zclorg}
\author{张朝龙}
\date{}
\title{线性映射的对偶的零空间和值域}
\hypersetup{
 pdfauthor={张朝龙},
 pdftitle={线性映射的对偶的零空间和值域},
 pdfkeywords={},
 pdfsubject={},
 pdfcreator={Emacs 25.0.50.1 (Org mode 9.0.5)}, 
 pdflang={English}}
\begin{document}

\maketitle
我们在 \href{duality.org}{对偶空间与对偶映射} 一节讨论了对偶空间和对偶映射的定义。对偶空间把线性泛函和一个特定的空间\(V\)联系起来,从对偶空间导出对偶基的概念。从一般的线性映射导出了对偶映射的概念,进而导出对偶映射的一些性质,这些性质和其对应的线性映射之间存在紧密的联系。

今天学习线性映射的对偶的零空间和值域。顾名思义,线性映射\(T\)的对偶\(T^{'}\)指对偶空间\(\mathcal{L}(W^{'},V^{'})\)中的映射,其零空间和值域按照零空间和值域的记号可以写为:\(nullT^{'},rangeT^{'}\)。显然对偶映射与其对应的线性映射之间存在紧密的联系,则对偶空间的零空间和值域也必然与\(nullT\)和\(rangeT\)之间存在紧密的联系。

\begin{definition}
对于\(U\subset V\),\(U\)的零化子(annihilator)\(U^{0}\)定义如下:\[U^{0} = \{\varphi\in V^{'}: \forall u\in U, \varphi(u) = 0\}\]
\end{definition}

从定义可以解读,零化子是线性泛函的集合,这个线性泛函是针对\(V\)的对偶空间。属于零化子的线性泛函具有\(\forall u\in U,\varphi(u) = 0\)的特性。

\begin{instance}
设\(U\)是\(\mathcal{P}(\mathbf{R})\)的用\(x^{2}\)乘以所有多项式所得到的子空间,若\(\varphi\)是\(\mathcal{P}(\mathbf{R})\)上由\(\varphi(p) = p^{'}(0)\)定义的线性泛函,则\(\varphi\in U^{0}\)
\end{instance}

对于\(U\subset V\),零化子\(U^{0}\)是\(V^{'}\)的子集。于是\(U^{0}\)依赖于包含\(U\)的向量空间,所以记号\(U_{V}^{0}\)或许更准确,这个记号告诉我们\(U\subset V\)且零化子是属于\(V^{0}\)的子集。

\begin{instance}
用\(e_{1},e_{2},e_{3},e_{4},e_{5}\)表示\(\mathbf{R}^{5}\)的标准基,用\(\varphi_{1},\varphi_{2},\varphi_{3},\varphi_{4},\varphi_{5}\)表示\((\mathbf{R}^{5})^{'}\)的对偶基。设:
\begin{equation}
\label{eq:8}
U = span(e_{1},e_{2}) = \{(x_{1},x_{2},0,0,0)\in \mathbf{R}^{5}:x_{1},x_{2}\in \mathbf{R}\}
\end{equation}
证明:\(U^{0}=span(\varphi_{3},\varphi_{4},\varphi_{5})\)
\end{instance}

\begin{proof}
因为\(\varphi_{1},\varphi_{2},\varphi_{3},\varphi_{4},\varphi_{5}\)是\((\mathbf{R}^{5})^{'}\)的对偶基,则这个对偶基是把\(\mathbf{R}^{5}\)中的向量\(x=(x_{1},x_{2},x_{3},x_{4},x_{5})\)变为对应坐标元素的基,即:
\(\varphi_{i}(x) =x_{i},\forall i\in \{1,2,3,4,5\}\)

设线性泛函\(\varphi\in span(\varphi_{3},\varphi_{4},\varphi_{5})\),则\(\exists c_{3},c_{4},c_{5}\),使得:\(\varphi = c_{3}\varphi_{3}+c_{4}\varphi_{4} + c_{5}\varphi_{5}\),显然\(\varphi\in (\mathbf{R}^{5})^{'}\),又因为\(U = span(e_{1},e_{2})\),则对于\(x=(x_{1},x_{2},0,0,0)\in U\),有:
\[\varphi(x) = c_{3}\varphi_{3}(x) + c_{4}\varphi_{4}(x) + c_{5}\varphi_{5}(x) = 0\]
所以\(\varphi\in U^{0}\),又由于\(\varphi\)的任意性,\(span(\varphi_{3},\varphi_{4},\varphi_{5})\subseteq U^{0}\)

接下来我们证明另一方面:证明\(U^{0}\in span(\varphi_{3},\varphi_{4},\varphi_{5})\)。

设\(\varphi\in U^{0}\),因为\(U^{0}\)是\((\mathbf{R}^{5})^{'}\)的子集,则\(\varphi\)可以表示成\((\mathbf{R}^{5})^{'}\)的基的线性组合,即:\(\exists c_{1},c_{2},c_{3},c_{4},c_{5}\)使得:
\(\varphi = c_{1}\varphi_{1} + c_{2}\varphi_{2} +\ldots + c_{5}\varphi_{5}\),因为\(\varphi\in U^{0}\),则对于\(u\in U\),有\(\varphi(u) = 0\)。因为\(U=span(e_{1},e_{2})\),\(\varphi(e_{1}) = 0, \varphi(e_{2}) = 0\),进而\(c_{1} = 0,c_{2} = 0\),所以:\[\varphi = c_{3}\varphi_{3} + c_{4}\varphi_{4} + c_{5}\varphi_{5}\] 即:\(U^{0}\subseteq span(\varphi_{3},\varphi_{4},\varphi_{5})\)


综上有:\(U^{0} = span(\varphi_{3},\varphi_{4},\varphi_{5})\)
\end{proof}

\begin{theorem}
设\(U\subset V\),则\(U^{0}\)是\(V^{'}\)的子空间。
\end{theorem}

\begin{proof}
证明这样的问题,我们可以从证明子空间的三点出发:1. 包含零元,2. 可加性,3. 齐次性。

\(U^{0}\)中包含线性泛函\(0\)是显然的。接下来我们证明可加性和齐次性。

设\(\varphi,\phi\in U^{0}\),则对于\(u\in U\),有:
\begin{equation}
\label{eq:1}
(\varphi + \phi)(u) = \varphi(u) + \phi(u) = 0
\end{equation}

另外对于\(\lambda\in \mathbf{F},\phi\in U^{0}\),则对于\((\lambda\phi)(u) = \lambda(\phi(u)) = \lambda 0 = 0\)

所以零化子\(U^{0}\)是\(V^{'}\)的子空间。
\end{proof}

对于零化子的维数有一个结论:
\begin{theorem}
设\(V\)是有限维的,\(U\)是\(V\)的子空间,则:\[\dim U + \dim U^{0} = \dim V \]
\end{theorem}

\begin{proof}
证明之前,明确一下\(U^{0}\),\(U^{0}\)是\(U\)零化子,零化子是线性泛函的集合,零化子里的线性泛函把\(\forall u\in U\)映射为\(0\)。

设\(i\in \mathcal{L}(U,V)\)是包含映射,定义如下:对\(u\in U\)有\(i(u) = u\),则\(i^{'}\)是\(V^{'}\)到\(U^{'}\)的线性映射。对\(i^{'}\)应用线性映射基本定理有:
\begin{equation}
\label{eq:2}
\dim range i^{'} +\dim null i^{'} = \dim V^{'} 
\end{equation}
而\(null i^{'} = U^{0}\),且\(\dim V^{'} = \dim V\),故上式变为:
\begin{equation}
\label{eq:3}
\dim range i^{'} + \dim U^{0} = \dim V
\end{equation}

若\(\phi \in U^{'}\),则\(\phi\)可以扩张为\(V\)上的线性泛函\(\psi\),\(i^{'}\)的定义表明\(i^{'}(\psi) = \phi\).所以\(\phi \in range i^{'}\),这表明\(range i^{'} = U^{'}\)。因此:\[\dim range i^{'} = \dim U^{'} = \dim U\]

综上原命题得证。
\end{proof}

这个命题的证明过程综合了好多个知识点,现在我们慢慢消化它。首先:从定义\(i(u) = u\)和\(i^{'}\)是从\(V^{'}\)到\(U^{'}\)的线性映射出发。我们知道\(i^{'}(\phi) = 0,\phi\in V^{'}\)意味着\(\phi\circ i = 0\),又因为\(i\)是包含映射,所以有:\(\phi\in U^{0}\)

另外对于\(i^{'}\)的定义,这个线性映射把一个线性泛函映射为另外一个线性泛函,要紧扣对偶映射的定义。

\begin{theorem}
设\(V\)和\(W\)都是有限维,\(T\in \mathcal{L}(V,W)\),则:
\begin{enumerate}
\item \(null T^{'} = (rangeT)^{0}\)
\item \(\dim nullT^{'} = \dim nullT  + \dim W - \dim V\)
\end{enumerate}
\end{theorem}

\begin{proof}
\begin{enumerate}
\item 首先假设\(\varphi \in nullT^{'}\),则\(0 = T^{'}(\varphi) = \varphi\circ T\),对于\(v\in V\),有:\[0 = (\varphi\circ T)(v) = \varphi(Tv)\]于是\(\varphi \in (rangeT)^{0}\),即\[nullT^{'}\subseteq (rangeT)^{0}\] 为了证明另外一个方面,设\(\varphi\in (rangeT)^{0}\),我们知道\((rangeT)^{0} =\{\phi\in W^{'}:\forall \omega\in rangeT, \phi(\omega) = 0\}\),假设\(\varphi\in (rangeT)^{0}\),我们要证明\(\varphi \in nullT^{'}\)。因为\(\varphi \in (rangeT)^{0}\),则有:\(\forall v \in rangeT, \varphi(Tv) = 0 = (\varphi \circ T)v\),显然有\(\varphi\circ T = 0 = T^{'}(\varphi)\),即,\[\varphi \in nullT^{'}\],即\[(rangeT)^{0} \subseteq nullT^{'}\]
\item 第二步的证明:
\end{enumerate}
\begin{eqnarray}
\label{eq:5}
\dim nullT^{'}&=&\dim (rangeT)^{0} \\
&=& \dim W - \dim rangeT \\
&=& \dim W - (\dim V - \dim nullT) \\
&=& \dim nullT + \dim W - \dim V
\end{eqnarray}
第一个等式直接利用第一步的结果,第二个等式利用零化子的维数公式,第三个等式利用了线性映射基本定理。 
\end{proof}

\begin{theorem}
设\(V\)和\(W\)是有限维的,\(T\in \mathcal{L}(V,W)\),则\(T\)是满的当且仅当\(T^{'}\)是单的。
\end{theorem}

\begin{proof}
我们之前有\(nullT^{'} = (rangeT)^{0}\),所以\(rangeT = W\)当且仅当\((rangeT)^{0} = \{0\}\),当且仅当\(nullT^{'} = \{0\}\),即\(T^{'}\)是单的。
\end{proof}

\begin{theorem}
设\(V\)和\(W\)都是有限维的,\(T\in \mathcal{L}(V,W)\),则:
\begin{enumerate}
\item \(\dim rangeT^{'} = \dim rangeT\)
\item \(rangeT^{'} = (nullT)^{0}\)
\end{enumerate}
\end{theorem}

\begin{proof}
首先我们证明第一个问题:
\begin{eqnarray}
\label{eq:6}
\dim rangeT^{'}&=& \dim  W^{'} - \dim nullT^{'} \\
&=& \dim W - \dim (rangeT)^{0} \\
&=& \dim rangeT
\end{eqnarray}

第一个等式是线性应设定里的直接使用。第二个等式是\(\dim W = \dim W^{'}\)和\(\dim nullT^{'} - \dim (rangeT)^{0}\)的实用。第三个等式是\(\dim U + \dim U^{0} = \dim V\)的直接使用。

然后我们证明第二个问题:
设\(\varphi\in rangeT^{'}\),由于\(rangeT^{'}\subseteq V^{'}\),则\(\varphi\in V^{'}\)。存在\(\psi\in W^{'}\),使得\(T^{'}(\psi) = \varphi\),设\(v\in nullT\),则有\(v \in V\),所以\(\varphi(v) = T^{'}(\psi)(v) = \psi\circ T(v) = 0\),所以\(\varphi \in (nullT)^{0}\),即\(rangeT^{'} \subseteq (nullT)^{0}\)

为了完成证明,我们需要证明\(\dim rangeT^{'} = \dim (nullT)^{0}\),注意:
\begin{eqnarray}
\label{eq:7}
\dim rangeT^{'}&=&\dim rangeT \\
&=& \dim V - \dim nullT \\
&=& \dim (nullT)^{0}
\end{eqnarray}
\end{proof}

\begin{theorem}
\(T\)是单的等价于\(T^{'}\)是满的。
\end{theorem}

\begin{proof}
映射\(T\in \mathcal{L}(V,W)\)是单的当且仅当\(nullT = \{0\}\),当且仅当\((nullT)^{0} = V^{'}\) (因为当\(nullT = \{0\}\)时,\(V^{'}\)中的任意一个线性泛函都可以把\(nullT\)中的元素映射为0) ,当且仅当\(\dim rangeT^{'} =\dim V^{'}\),即\(rangeT^{'} = V^{'}\)
\end{proof}
\end{document}
