% Intended LaTeX compiler: pdflatex
\documentclass[10pt,a4paper,UTF8]{article}
\usepackage{zclorg}
\author{张朝龙}
\date{}
\title{对偶空间与对偶映射}
\hypersetup{
 pdfauthor={张朝龙},
 pdftitle={对偶空间与对偶映射},
 pdfkeywords={},
 pdfsubject={},
 pdfcreator={Emacs 25.0.50.1 (Org mode 9.0.5)}, 
 pdflang={English}}
\begin{document}

\maketitle
在线性代数中,映射到标量域\(\mathbf{F}\)的线性映射具有非常重要的作用。
\begin{definition}
\(V\)上的线性泛函是从\(V\)到\(\mathbf{F}\)的线性映射。也就是说线性泛函是\(\mathcal{L}(V,\mathbf{F})\) 中的元素。
\end{definition}

\begin{instance}
\begin{enumerate}
\item 定义\(\phi: \mathbf{R}^{3} \rightarrow \mathbf{R}\)为\(\phi(x,y,z) = 4x-5y + 2z\),则\(\phi\)是\(\mathbf{R}^{3}\)上的线性泛函。
\item 取定\((c_{1},\ldots ,c_{n})\in \mathbf{F}^{n}\),定义\(\phi : \mathbf{F}^{n} \rightarrow \mathbf{F}\)为:\(\phi(x_{1},\ldots ,x_{n}) = c_{1}x_{x} + \ldots c_{n}x_{n}\),则\(\phi\)是\(\mathbf{F}^{n}\)上的线性泛函。
\item 定义\(\phi: \mathcal{P}( \mathbf{R}) \rightarrow \mathbf{R}\)为\(\phi(p) = 2p^{''}(5) + 7p(4)\),则\(\phi\)是\(\mathcal{P}( \mathbf{R})\)上的线性泛函。
\item 定义\(\phi: \mathcal{P}(\mathbf{R}) \rightarrow \mathbf{R}\)为\(\phi(p)=\int_{0}^{1} p(x)dx\),则\(\phi\)是\(\mathcal{P}(\mathbf{R})\)上的线性泛函。
\end{enumerate}
\end{instance}

\begin{definition}
\(V\)上的所有线性泛函构成的向量空间称为\(V\)的对偶空间,记为\(V^{'}\)。也就是说,\(V^{'} = \mathcal{L}(V, \mathbf{F})\)
\end{definition}

\begin{theorem}
\(\dim V^{'} = \dim V\)
\end{theorem}

\begin{proof}
我们之前 3.61 证明过:\(\dim(\mathcal{L}(V,W)) = (\dim V)(\dim W)\)。

对于这个命题因为\(V^{'} = \mathcal{L}(V,F)\) ,所以\(\dim(V^{'}) = \dim(V)\dim(\mathbf{F})\),又因为\(\dim \mathbf{F} = 1\) .

这里\(V^{'}\)是线性泛函的集合,(线性泛函都是从\(V\)到\(\mathbf{F}\)的映射。)
\end{proof}

\begin{definition}
设\(v_{1},\ldots ,v_{n}\)是\(V\)的基,则\(v_{1},\ldots ,v_{n}\)的对偶基是\(V^{'}\)中的元素组\(\varphi_{1}, \ldots ,\varphi_{n}\),其中每个\(\varphi_{j}\)都是\(V\)上的线性泛函,满足:
\begin{equation}
\label{eq:1}
\varphi_{j}(v_{k}) = 
\begin{cases}
1, & k=j \\
0, & k\neq j 
\end{cases}
\end{equation}
\end{definition}

\begin{instance}
求\(\mathbf{F}^{n}\)的标准基\(e_{1},\ldots ,e_{n}\)的对偶基。

对于\(1\leq j \leq n\),定义\(\varphi_{j}\)是\(\mathbf{F}^{n}\)上的线性泛函,满足\(\forall (x_{1},\ldots ,x_{n})\in \mathbf{F}^{n}\):
\begin{equation}
\label{eq:2}
\varphi_{j}(x_{1},\ldots ,x_{n}) = x^{j}
\end{equation}

显然:
\begin{equation}
\label{eq:3}
\varphi_{j}(v_{k}) = 
\begin{cases}
1, & k=j \\
0, & k\neq j 
\end{cases}
\end{equation}
于是\(\varphi_{1},\ldots ,\varphi_{n}\)是\(\mathbf{F}^{n}\)的标准基\(e_{1},\ldots ,e_{n}\)的对偶基。

从对偶基的定义可以看出对偶基与\(V\)的基紧密相关,由于对偶基是\(V^{'}\)中满足特定条件的线性映射,根据定义,对偶基是把\(V\)的基的各个元素映射称\(1\)或者\(0\)的线性泛函的集合。注意对偶基是把\(V\)中的基映射为\(\mathbf{F}\)中的\(1\)而不是其他元素,所以可以想见这个对偶基在以后有很多特殊的应用。
\end{instance}

\begin{theorem}
设\(V\)是有限维的,则\(V\)的一个基的对偶基是\(V^{'}\)的基。
\end{theorem}

\begin{proof}
还是从定义出发逐一解读这个命题的关键元素。

首先\(V\)是有限维的,说明\(V\)的维度有限。\(V\)的一个基的对偶基是\(V^{'}\)中的元素,这些元素是线性泛函,这些线性泛函把\(V\)的基映射称\(\mathbf{F}\)中的\(1\)或者\(0\),另外注意:\(1\)是\(\mathbf{F}\)中的乘法单位元,\(0\)是\(\mathbf{F}\)中的加法零元。

所以我们假设\(v_{1},\ldots ,v_{n}\)是\(V\)的基,则\(V^{'}\)的对偶基也有\(n\)个元素,假设为\(\varphi_{1},\ldots ,\varphi_{n}\)。我们接下来要证明\(\varphi_{1},\ldots ,\varphi_{n}\)是线性无关且张成\(V^{'}\)。

为了证明\(\varphi_{1},\ldots ,\varphi_{n}\)是线性独立的,令:
\begin{equation}
\label{eq:4}
0 = a_{1}\varphi_{1} + \ldots + a_{n}\varphi_{n}
\end{equation}
我们只要得到\(a_{i},\forall i\)即可。注意上式左端的\(0\)是对偶空间中的\(0\)元素,是一个线性泛函。

对上式两端我们作用于\(v_{i}\),显然有\(0v_{i} = 0 = a_{i}\varphi_{i}(v_{i})\) . 根据对偶基的定义,我们有\(a_{i}\varphi_{i}(v_{i}) = a_{i}, a_{j}\varphi_{j}(v_{i}) = 0, \forall j\neq i\).

所以\(a_{i} = 0, \forall i\)

又因为,我们之前有\(\dim V^{'} = \dim V\)。而\(\dim span(\varphi_{1},\ldots ,\varphi_{n}) = n\),所以\(\dim V^{'} = n\)。

所以\(\varphi_{1},\ldots ,\varphi_{n}\)是\(V^{'}\)的一组基(若\(V^{'}\)是有限维的,则\(V^{'}\)中每个长度为\(\dim V^{'}\) 的线性无关向量组都是\(V^{'}\)的基)。
\end{proof}

\begin{definition}
若\(T\in \mathcal{L}(V,W)\),则\(T\)的对偶映射是线性映射\(T^{'} \in \mathcal{L}(W^{'},V^{'})\):对于\(\varphi \in W^{'}\), \(T^{'}(\varphi) = \varphi \circ T\)
\end{definition}

注意这里的\(W^{'},V^{'}\)分别是\(W,V\)上的所有线性泛函构成的向量空间,即\(W,V\)的对偶空间。\(\varphi\in W^{'}\)表明\(\varphi\)是从\(W\)到\(\mathbf{F}\)的线性映射。


如果\(T\in \mathcal{L}(V,W)\), \(\varphi \in W^{'}\),那么\(T^{'}(\varphi)\)被定义为线性映射\(\varphi\)与\(T\)的复合。于是,由于\(T\)是从\(V\)到\(W\)的线性映射,而\(\varphi\)是从\(W\)到\(\mathbf{F}\)的线性泛函。所以\(\varphi\circ T\)是从\(V\)到\(\mathbf{F}\)的线性泛函,即 \(T^{'}(\varphi)\)的确是\(V\)到\(\mathbf{F}\)的线性映射。也就是说,\(T^{'}(\varphi)\in V^{'}\)

验证\(T^{'}\)是\(W^{'}\)到\(V^{'}\)的线性映射:
\begin{enumerate}
\item 若\(\varphi,\phi \in W^{'}\),则\(T^{'}(\phi + \varphi) = ( \phi + \varphi )\circ T = \phi \circ T + \varphi \circ T = T^{'}(\phi) + T^{'}(\varphi)\)
\item 若\(\lambda \in \mathbf{F}, \varphi\in W^{'}\),则\(T^{'}(\lambda\varphi) = (\lambda\varphi)\circ T = \lambda(\varphi\circ T) = \lambda T^{'}(\varphi)\)
\end{enumerate}

在下面的例子中,\(^{'}\)有两种毫不相干的意义:\(D^{'}\)表示线性映射\(D\)的对偶映射,\(p^{'}\)则表示多项式\(p\)的导数。

\begin{instance}
定义:\(D: \mathcal{P}(\mathbf{R}) \rightarrow \mathcal{P}(\mathbf{R})\)为\(Dp = p^{'}\)
\begin{enumerate}
\item 设\(\varphi\)是\(\mathcal{P}(\mathbf{R})\)上由\(\varphi(p) = p(3)\)定义的线性泛函。则\(D^{'}(\varphi)\)是\(\mathcal{P}(\mathbf{R})\)上如下定义的线性泛函:\[(D^{'}(\varphi))(p) =(\varphi\circ D)(p) = \varphi(Dp) = \varphi(p^{'}) = p^{'}(3) \]即:\(D^{'}(\varphi)\)是\(\mathcal{P}(\mathbf{R})\)上将\(p\)变成\(p^{'}(3)\)的线性泛函。
\item 设\(\varphi\)是\(\mathcal{P}(\mathbf{R})\)上由\(\varphi(p) = \int_{0}^{1}p(x)dx\)定义的线性泛函。则\(D^{'}(\varphi)\)是\(\mathcal{P}(\mathbf{R})\)上如下定义的线性泛函:\[(D^{'}(\varphi)(p)) = (\varphi\circ D)(p) = \varphi(D(p)) = \varphi(p^{'}) = \int_{0}^{1} p^{'}(x)dx = p(1) - p(0)\]即:\(D^{'}(\varphi)\)是\(\mathcal{P}(\mathbf{R})\)上将\(p\)变为\(p(1)-p(0)\)的线性泛函。
\end{enumerate}
\end{instance}

对偶映射的代数性质:
\begin{enumerate}
\item 对所有\(S,T\in \mathcal{L}(V,W)\)有\((S+T)^{'} = S^{'} + T^{'}\)
\item 对所有\(\lambda \in \mathbf{F}\)和所有\(T\in \mathcal{L}(V,W)\),有\((\lambda T)^{'} = \lambda T^{'}\)
\item 对所有\(T\in \mathcal{L}(U,V)\)和所有\(S\in \mathcal{L}(V,W)\),有\((ST)^{'} = T^{'}S^{'}\)
\end{enumerate}

\begin{proof}
对于第一条,根据对偶映射的定义 (\textbf{若\(T\in \mathcal{L}(V,W)\),则\(T\)的对偶映射是线性映射\(T^{'}\in \mathcal{L}(W^{'},V^{'})\):对于\(\varphi\in W^{'}, T^{'}(\varphi)= \varphi\circ T\)}),对于\(\varphi\in W^{'}\),有:
\begin{equation}
\label{eq:5}
(S^{'} + T^{'})(\varphi) = \varphi \circ (S + T) = \varphi \circ S + \varphi \circ T = S^{'})(\varphi) + T^{'}(\varphi)  = (S^{'} + T^{'})(\varphi)
\end{equation}

对于第二条,同样根据对偶映射的定义(若\(T\in \mathcal{L}(V,W)\),则\(T\)的对偶映射\(T^{'}\in \mathcal{L}(W^{'},V^{'})\),对于\(\varphi\in W^{'}\),有\(T^{'}(\varphi) = \varphi\circ T\)) ,对于\(\varphi \in W^{'}\),有:
\begin{equation}
\label{eq:6}
(\lambda T)^{'} (\varphi) = \varphi \circ (\lambda T) =\lambda (\varphi \circ T) = \lambda (T^{'}(\varphi))
\end{equation}

对于第三条:

假设有\(\varphi\in W^{'}\),则有:
\begin{equation}
\label{eq:7}
(ST)^{'}(\varphi) = \varphi \circ ST = (\varphi \circ S )\circ T = T^{'}(\varphi \circ S) = T^{'}(S^{'}(\varphi)) = T^{'}S^{'}(\varphi)
\end{equation}

使用对偶映射的定义推导出第一个等号,使用映射的结合性推出第二个等号,使用对偶映射的定义推导出第三个等号,使用对偶映射的结合性对导出第四个等号,使用映射的结合性推出第五个等号。
\end{proof}
\end{document}
