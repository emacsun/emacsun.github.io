\documentclass[10pt,a4paper,UTF8]{article}
\usepackage{zclorg}
\author{Eastern(ZCL)}
\date{2015年12月27日}
\title{深度学习的艺术}
\hypersetup{
 pdfauthor={Eastern(ZCL)},
 pdftitle={深度学习的艺术},
 pdfkeywords={study},
 pdfsubject={},
 pdfcreator={Emacs 25.0.50.1 (Org mode 8.3.2)}, 
 pdflang={English}}
\begin{document}

\maketitle
\tableofcontents


\section{引言}
\label{sec:orgheadline1}


这篇博文是对采铜《深度学习的艺术》和《开放智力》的总结,也是对我学习方法论的一次改进。本文也以“深度学习的艺术”为题,一是向采铜致敬,二来是宣告我现在学习方法对过去的不同。至于掌握深度学习的艺术,我还有很多东西要学。

\section{深度学习}
\label{sec:orgheadline2}


所谓深度学习,就是对学习对象不仅看表面,更要深入挖掘,直到获得自己之前没有达到的理解层次,甚至看到别人也不曾看见的东西。我之所以要努力去学习深度学习方面的东西,就是工作之后感觉对教科书上的东西理解过于肤浅。明明我已经花费了那么多的时间去学习,最后的结果还是不尽如人意,一定是方法论出了问题。在学习过程中,我逐渐明白:之前的学习更多的是对资料的记忆,而非对知识的理解。从资料转化为知识,并不是想当然的事情。采铜从“如何成为高段位的学习者”这个问题出发,从古今中外的先贤的智慧中萃取养分,结合心理学和教育学的一般规律,就“深度学习”做了系统的阐述。其中诸多观点,令人耳目一新。

“深度学习”可以从四个方面展开:提问,解码,操练,融合。这四个部分展示了深度学习的每一个侧面,既相互独立有互相关联。特别要说明的是,并不是对所有的学习资料都要用本文所说的深度学习方法来学习,就像我在《主动阅读》里提到的一样,主动阅读和深度学习都只适合一些与我们的目标强相关的资料。对于一些娱乐和咨询类的资料,达到目的就可以。精力有限,学海无涯,学习亦需谨慎,切莫走火入魔。与我们目标强相关的资料也只有一些信息密度高,微言大义,博大精深的值得我们用深度学习的方法去琢磨,对于一些粗浅的材料,浅尝就好。至于在具体操作中,如何分辨学习材料是否值得深度学习,这是一门“艺术”。玄之又玄,妙之又妙,行而之上,不可与外人道也(其实我也不知道)。

\section{提问}
\label{sec:orgheadline3}


采铜的《深度学习的艺术》和《开放的智力》都是从问题出发,搜集资料,萃取,然后成书。《深度学习的艺术》始于问题“如何成为高段位的学习者”,就这一问题,作者旁征博引,查阅各方资料,反复推敲,迭代优化答案,最终提炼出来深度学习的四个步骤。《开放的智力》更是作者在知乎上对一系列问题答案的合集。

正是那些长期困扰人们的问题推动了人类智慧的进步。困扰了人们500年的费马大定理终于在1994年告破,在解决这个问题的过程中,诸多理论创新被提出。人们就“薛定谔的猫是死是活”展开了量子领域的大讨论。就“光到底具有波动性还是粒子性”这个问题,物理学家进行了三次论战。“我是谁”这样的经典哲学问题,刺激者哲学家们不停的探索。我们不一定非要成为伟大的数学家,物理学家或者哲学家:一个好的问题依然可以成为我们最好的导师;好问题激起的对答案的渴望是我们前进的不竭动力。

想象此前我的学习流程,太过鲁莽。对于一些学习资料,常常展开地毯式的轰炸:首先制定周密的计划,仔细的阅读材料,认真的做笔记,复习笔记。到头来,却不知道为什么要这么做?不知道要解决什么问题。最后,那些笔记被尘封,曾经弄明白的知识点因为没有及时的嵌入已有的知识体系而被遗忘。究其原因,还是没有问题的指引。

波利亚在《怎样解题》中也提到:“没有任何一个问题是彻底完成的。总还会有一些事情做;在经过充分的研究和猜测之后,我们可以将任何解题方法加以改进;而且无论如何,我们总可以深化我们对答案的理解。”尤其是那些值得长期探索的问题,不仅可以充分调动我们的潜意识帮助我们思考(苯环结构的发现就是如此),更可以充分调动我们已有的知识点,把这些零散的“珍珠”串起来,变成美丽的项链,就像乔布斯在斯坦福毕业典礼上演讲时说的那样。一个好的长期问题的价值半衰期是很长的,在相当长的时间内,我们都将收益于这个问题。这些问题让我们成为“建构者”,因为我们不仅在学习知识,还在建构答案,在努力回答问题的过程中,我们筛选,评判和整合新知识和就只是,并把他们融合成一个自给的整体;一个好的问题让我们成为一个勇敢的“猎手”,为了达到目的,我们有勇气去涉猎从未到达的猎区,我们的知识面和视野就是这么拓宽的。

\section{解码}
\label{sec:orgheadline4}



通常我们对材料只进行了最表浅的加工,没有深入挖掘,更不会去下一番功夫解码。解码不同于理解,理解通常只涉及字面意义的解读,常以自动化的方式进行,并且理解应遵从本意,不可擅自曲解。而解码则是在理解的基础上进行更为深入和主动的探索过程,有可能涉及材料意外的知识点的链接。在这个过程中,不同的观点被调动,个人观点形成。一个人是否博学不在于记忆了多少知识点,在于可以调动多大范围的知识点,在于他的知识点之间链接是否丰富是否鲁棒。一个好的解码者可以调动大范围的知识点对当前的材料进行多角度的审视。解码与提问有着一脉相承的关系,不同的问题就会导致不同的解码角度。

采铜用一个“玩具小黄鸭”的例子来阐述解码的三个层次:
\begin{enumerate}
\item 对于小朋友,他关心的是小黄鸭唱的儿歌。也就是它说了什么。
\item 对于爸妈,他关心的是小黄鸭是什么,家长会把小黄鸭定义为玩具,然后对其娱乐性,教育性,安全性和性价比做一番评判。家长们关心它是什么。
\item 对于玩具工程师,他关心的是小黄鸭是怎么做出来的,使用了哪些技术。也就是它是怎么实现的。
\end{enumerate}

当然,解码远非仅仅以上三个层次。对于学习者而言,是否善于解码决定了我们对知识掌握的效果,也决定了我们是什么样的人才。教育心理学家把在某一领域有专长的人分为“常规型专长”和“适应性专长”两类,其中具有常规型专长的人具有一个基本固定的知识系统,可以以很高的效率把接触到的信息材料按照已有的框架进行分析,而具有适应型专长的人则可以不断进化扩充他们的核心能力,可扩展专业知识的广度和深度来迎合需求和兴趣的增长。显然,在当代瞬息万变的社会中,后者的生命力和适应能力更强。怀特海在《思维方式》中提到:理解的推进有两种,一种是把细节集合与既定的模式,一种是发现强调新细节的新模式。他接力推崇第二种,反对第一种。其实第二种模式就是适应型专长的人具有的思维模式。

那么,如何才能高效的对已有材料进行解码呢?采铜给出了三条建议:

\begin{enumerate}
\item \textbf{不仅要去寻找结论,还要去寻找过程。} 阅读一本书或者欣赏一部电影,不能仅仅关注结论,还要留心整个过程是如何演绎的。我们要像福尔摩斯一样,看到犯罪现场,就可以把犯罪现场在头脑中还原。

\item \textbf{不只要做归纳,还要做延展} 这与“先把书读薄,再把书读厚”不谋而合。通常,我们都比较善于归纳,抽象。阅读完一本书,我们往往可以用一句话高度的抽象本书的要义。但是,高度抽象的结论必然带来信息的大量丢失。我们还应具备延展结论的能力,就像树一样,从树干到树枝,把结论分解成多个分论点,然后再分。事实上,我们的知识树就是这样建立起来的,各个知识点之间互相链接,我们才好调用。分散的知识点,迟早会变成枯枝败叶。

\item \textbf{不仅要比较相似,还要寻找不同} 我们总倾向于把新知识与自己已有的知识体系做对比,然后找到相同,用已有的知识体系去理解,这样感觉比较安全。一旦不同点太多,我们就会觉得难以与已有的知识体系发生联系。就会觉得“很难”。实际上,正是这些不同点扩展了我们的知识体系,而那些相同点仅仅验证和加强了已有的知识体系。一个适应型专长的人应该有接纳不同的勇气和智慧。
\end{enumerate}

\section{操练}
\label{sec:orgheadline5}


这个最好理解,就是要调用学到的知识点。用的多了知识自然就内化,成为身体的一部分。不止一处的心理学和教育学研究表明,那些在项目中学习到的知识比在传统课堂上学到的知识印象要深刻的多。攻读硕士博士期间,老师通常会安排不少项目(不要考虑老师是处于什么目的),这是很好的锻炼机会。操练的方式有很多,采铜也给出了不少建议,包括:游戏式操练,设计式操练,写作式操练。我执行最多的是写作式操练,这也是我此刻写博客的原因。

写作是一种典型的知识重构活动,不仅是对已有知识的提取和调用,更是对这些知识点的重构。在阅读过程中获得的知识往往比较肤浅,甚至有时候大脑还会欺骗我们认为学会了。在写作过程中,所有的问题都会暴露出来,因为我们必须对这个知识点有非常深刻的认识,才能用自己的语言总结出来。另外,通过阅读获得的知识点通常比较零散,而在写作过程中,我们必须去比较和分析它们,并试图建立联系。刘未鹏在《为什么你应该从现在开始写博客》中对写作的好处解释的理据服( 有理有据令人信服,简称理据服)。除了重构知识点,写博客还可以广交好友,与志同道合的人沟通交流。

\section{融合}
\label{sec:orgheadline6}


融合就是就知识和新知识发生联系,最终实现知识结构的更新重构。通过对一本教材的解码,我们的大脑中得到的是这本教材的知识树,在这种结构下,每一片叶子都是分散而独立的。通过联系其他材料的知识,我们才构建知识网络。对于一个现实问题,往往要联系不同领域的知识才能得以解决。人类智慧发展到今天,培养了大量的专才,而那些伟大的创新往往是跨学科的灵光闪现。怀特海在《教育的目的》中强烈建议:根除各科目之间那种致命的分离状态,因为它扼杀了现代课程的生命力。教育只有一个主题,那就是五彩缤纷的生活。但我们没有向学生展现生活这个独特的统一体。来自查理芒格的抨击则更激烈,他把只会用单一学科思考的人成为“铁锤人”,这种人看见任何对象都会把它当成“钉子”。狭隘的观点限制了他基于场景去分析事实本身的能力。只有那些具有适应型专长的人才会不断的扩充自己的知识领域,避免思维上的偏差和狭隘。查理芒格认为只有多学科模型的方法才能产生爆炸性的合力效应,让人获得不同寻常的智慧。

\section{我的反思}
\label{sec:orgheadline7}


虽然已经获得博士学位,但是对于学习方法的掌握可以说相当原始。对于认知科学的学习虽然略知皮毛,但是系统的阅读这方面的书籍后才发现原来自己在黑暗中的摸索是多么的可怜。深度学习的四个环节都做得不是很好:总是等待导师提问,自己反省不足,甚至课题题目和博士开题报告都是导师指定,这表明好奇心的严重匮乏,更别说提出比较好的问题;对于研究课题相关材料的解码也存在缺陷,比较擅长把新鲜事物归类于已知知识体系,却往往忽略了差异的存在,很多时候,尤其做研究,差异意味着可能的创新;操练也不是很勤快。不过现在开始写博客,算是亡羊补牢;等到工作中才发现,这里的问题在教科书里找不到答案,必须融合多学科的知识才有可能得到答案。

所幸,现在我已经意识到问题的存在,改变正在发生。
\end{document}
