\documentclass[10pt,a4paper,UTF8]{article}
\usepackage{zclorg}
\author{zcl.space}
\date{}
\title{Matlab与C混合编程API之mxCreateStructMatrix}
\hypersetup{
 pdfauthor={zcl.space},
 pdftitle={Matlab与C混合编程API之mxCreateStructMatrix},
 pdfkeywords={matlab communication simulation C},
 pdfsubject={mxCreateNumericMatrix创建结构体数组},
 pdfcreator={Emacs 25.0.50.1 (Org mode 8.3.2)}, 
 pdflang={English}}
\begin{document}

\maketitle
\tableofcontents



\section{引言}
\label{sec:orgheadline1}


在matlab中创建和操作结构体非常便捷,mathwork公司把这种便捷延伸到了桥梁函数中。matlab为桥梁函数定义了创建结构体的API: \texttt{mxCreateStructMatrix} 。
\section{调用语法}
\label{sec:orgheadline2}


\texttt{mxCreateStructMatrix} 的调用语法如下
\lstset{language=C,label= ,caption= ,captionpos=b,numbers=none}
\begin{lstlisting}
#include "matrix.h"
mxArray *mxCreateStructMatrix(mwSize m, mwSize n,
     int nfields, const char **fieldnames);
\end{lstlisting}

输入参数表如下:
\begin{table}[htb]
\caption{\label{tab:orgtable1}
\texttt{mxCreateStructMatrix} 输入参数对照表}
\centering
\begin{tabular}{ll}
\hline
参数名 & 描述\\
\hline
m & 结构体矩阵的行数\\
n & 结构体矩阵的列数\\
nfields & 结构体矩阵中每个结构体域的个数\\
fieldnames & 结构体矩阵中每个结构体的域名\\
\hline
\end{tabular}
\end{table}

\texttt{mxCreateStructMatrix} 的返回值是一个指向 \texttt{mxArray} 的指针。我们还是通过一个例子来说明 \texttt{mxCreateStructMatrix} 的使用。

\section{一个例子}
\label{sec:orgheadline3}


这个例子实现了电话本功能,是目前为止用到的最复杂的例子。
\lstset{language=C,label= ,caption= ,captionpos=b,numbers=left}
\begin{lstlisting}
#include "mex.h"
#include "string.h"

#define MAXCHARS 80   /* max length of string contained in
					 each field */

/*  the gateway routine.  */
void mexFunction( int nlhs, mxArray *plhs[],
		  int nrhs, const mxArray *prhs[] )
{
     /* pointers to field names */
    const char **fnames;
    const mwSize *dims;
    mxArray    *tmp, *fout;
    char       *pdata=NULL;
    int        ifield, nfields;
    mxClassID  *classIDflags;
    mwIndex    jstruct;
    mwSize     NStructElems;
    mwSize     ndim;

    /* check proper input and output */
    if(nrhs!=1)
	mexErrMsgIdAndTxt(
		"MATLAB:phonebook:invalidNumInputs",
		"One input required.");
    else if(nlhs > 1)
	mexErrMsgIdAndTxt(
		"MATLAB:phonebook:maxlhs",
		"Too many output arguments.");
    else if(!mxIsStruct(prhs[0]))
	mexErrMsgIdAndTxt(
		"MATLAB:phonebook:inputNotStruct",
		"Input must be a structure.");
    /* get input arguments */
    nfields = mxGetNumberOfFields(prhs[0]);
    NStructElems = mxGetNumberOfElements(prhs[0]);
    /* allocate memory  for storing classIDflags */
    classIDflags = mxCalloc(nfields, sizeof(mxClassID));

    /* check empty field, proper data type,
     * and data type consistency;
     * and get classID for each field. */
    for(ifield=0; ifield<nfields; ifield++) {
	for(jstruct = 0; jstruct < NStructElems; jstruct++) {
	    tmp = mxGetFieldByNumber(prhs[0], jstruct, ifield);
	    if(tmp == NULL) {
		mexPrintf("%s%d\t%s%d\n", "FIELD: ",
			   ifield+1, "STRUCT INDEX :", jstruct+1);
		mexErrMsgIdAndTxt( "MATLAB:phonebook:fieldEmpty",
			"Above field is empty!");
	    }
	    if(jstruct==0) {
		if( (!mxIsChar(tmp) && !mxIsNumeric(tmp))
		     || mxIsSparse(tmp)) {
		    mexPrintf("%s%d\t%s%d\n", "FIELD: ", ifield+1,
			      "STRUCT INDEX :", jstruct+1);
		    mexErrMsgIdAndTxt( "MATLAB:phonebook:invalidField",
			    "Above field must have either string or
			     numeric non-sparse data.");
		}
		classIDflags[ifield]=mxGetClassID(tmp);
	    } else {
		if (mxGetClassID(tmp) != classIDflags[ifield]) {
		    mexPrintf("%s%d\t%s%d\n", "FIELD: ", ifield+1,
		    "STRUCT INDEX :", jstruct+1);
		    mexErrMsgIdAndTxt( "MATLAB:phonebook:invalidFieldType",
			    "Inconsistent data type in above field!");
		} else if(!mxIsChar(tmp) &&
			((mxIsComplex(tmp) || mxGetNumberOfElements(tmp)!=1))){
		    mexPrintf("%s%d\t%s%d\n", "FIELD: ", ifield+1,
		    "STRUCT INDEX :", jstruct+1);
		    mexErrMsgIdAndTxt( "MATLAB:phonebook:fieldNotRealScalar",
			    "Numeric data in above field must be scalar
			    and noncomplex!");
		}
	    }
	}
    }

    /* allocate memory  for storing pointers */
    fnames = mxCalloc(nfields, sizeof(*fnames));
    /* get field name pointers */
    for (ifield=0; ifield< nfields; ifield++){
	fnames[ifield] = mxGetFieldNameByNumber(prhs[0],ifield);
    }
    /* create a 1x1 struct matrix for output  */
    plhs[0] = mxCreateStructMatrix(1, 1, nfields, fnames);
    mxFree((void *)fnames);
    ndim = mxGetNumberOfDimensions(prhs[0]);
    dims = mxGetDimensions(prhs[0]);
    for(ifield=0; ifield<nfields; ifield++) {
	/* create cell/numeric array */
	if(classIDflags[ifield] == mxCHAR_CLASS) {
	    fout = mxCreateCellArray(ndim, dims);
	}else {
	    fout = mxCreateNumericArray(ndim, dims,
		   classIDflags[ifield], mxREAL);
	    pdata = mxGetData(fout);
	}
	/* copy data from input structure array */
	for (jstruct=0; jstruct<NStructElems; jstruct++) {
	    tmp = mxGetFieldByNumber(prhs[0],jstruct,ifield);
	    if( mxIsChar(tmp)) {
		mxSetCell(fout, jstruct, mxDuplicateArray(tmp));
	    }else {
		mwSize     sizebuf;
		sizebuf = mxGetElementSize(tmp);
		memcpy(pdata, mxGetData(tmp), sizebuf);
		pdata += sizebuf;
	    }
	}
	/* set each field in output structure */
	mxSetFieldByNumber(plhs[0], 0, ifield, fout);
    }
    mxFree(classIDflags);
    return;
}
\end{lstlisting}
代码中使用 \texttt{mxGetNumOfFields} 获得输入的结构体矩阵中每个结构体域的个数,使用 \texttt{mxGetNumberOfElements} 来获得输入结构体矩阵中结构体的个数。

\lstset{language=C,label= ,caption= ,captionpos=b,numbers=none}
\begin{lstlisting}
plhs[0] = mxCreateStructMatrix(1, 1, nfields, fnames);
\end{lstlisting}
创建一个结构体保存输出。

\lstset{language=C,label= ,caption= ,captionpos=b,numbers=none}
\begin{lstlisting}
mxSetFieldByNumber(plhs[0], 0, ifield, fout);
\end{lstlisting}
为输出结构体的每个域赋值。

假设我们的输入是
\begin{verbatim}
a.a =1;
a.b =4;
a.c = 'hello world'
\end{verbatim}

调用 \texttt{phonebook}
\begin{verbatim}
n=phonebook(a);
\end{verbatim}
则 \texttt{n} 的值为:
\begin{verbatim}
n.a =1;
n.b =4;
n.c ='hello world'
\end{verbatim}
\section{尾声}
\label{sec:orgheadline4}


\texttt{mxCreateStructMatrix} 是matlab中创建结构体的API,方便了matlab和C之间结构体类型数据的传递。
\end{document}
