\documentclass[10pt,a4paper,UTF8]{article}
\usepackage{zclorg}
\author{zcl.space}
\date{}
\title{Matlab与C混合编程之二:C/C++ MEX API}
\hypersetup{
 pdfauthor={zcl.space},
 pdftitle={Matlab与C混合编程之二:C/C++ MEX API},
 pdfkeywords={matlab communication simulation C},
 pdfsubject={本文描述LTE-FDD和TDD的频带资源},
 pdfcreator={Emacs 25.0.50.1 (Org mode 8.3.2)}, 
 pdflang={English}}
\begin{document}

\maketitle\xiaosihao
\tableofcontents\newpage\newpage




\section{引言}
\label{sec:orgheadline1}


为方便matlab和C混合编程,matlab定义了一些可以在C代码中使用的数据类型和函数。本文首先给出这些数据类型和函数,然后举例演示这些数据类型和函数的使用过程。由于各个API之间相对独立,在阅读本文的过程中,读者随时可以通过目录按钮跳转到自己感兴趣的内容。也可以通过留言板向我提问(我的主页地址是 \texttt{zcl.space} )。

\section{matlab提供的数据类型}
\label{sec:orgheadline2}



为了方便编写能够在matlab中调用的C代码(通过MEX文件调用),matlab定义了一些数据类型。在编写C代码过程中,程序员使用这些数据类型就像使用C语言的 \texttt{int} \texttt{char} 那样简单。表\ref{tab:orgtable1}给出了matlab提供的数据类型以及简单的描述。

\begin{table}[htb]
\caption{\label{tab:orgtable1}
matlab提供的数据类型}
\centering
\begin{tabular}{ll}
\hline
数据类型 & 描述\\
\hline
\texttt{mxArray} & matlab array,是的在C中可以定义matlab array\\
\texttt{mwSize} & 专门为描述matlab的矩阵大小定义的类型,主要是为了方便代码跨平台,大多数情况 \texttt{int} 也可以\\
\texttt{mxIndex} & 为索引变量定义的类型,与 \texttt{mwSize} 差不多\\
\texttt{mwSignedIndex} & 有符号整型,主要为描述矩阵大小定义\\
\texttt{mwChar} & 为 string 数组定义的类型\\
\texttt{mxLogical} & 为 logical 数组定义的类型\\
\texttt{mxClassID} & \\
\texttt{mxComplexity} & 指示数组元素是否包含复数的旗标(flag )\\
\texttt{mxGetEps} & EPS\\
\texttt{mxGetInf} & 无穷大\\
\texttt{mxGetNaN} & NaN\\
\hline
\end{tabular}
\end{table}
\section{matlab提供的函数API}
\label{sec:orgheadline7}


matlab为C编程提供了丰富的API。根据其功能,这些API主要可以分为以下四类:
\begin{enumerate}
\item 创建或者删除数组
\item 验证数据有效性
\item 存取数据
\item 数据类型转换
\end{enumerate}

我们分四个小节简要列出这些函数名称及其功能。

\subsection{创建或者删除数组}
\label{sec:orgheadline3}


这一类API方便程序员在C代码中创建特定类型的数组,并为其分配或者释放内存。表\ref{tab:orgtable2}给出了这类API的名称和相关定义。
\begin{table}[htb]
\caption{\label{tab:orgtable2}
创建删除数组API}
\centering
\begin{tabular}{ll}
\hline
名称 & 功能\\
\hline
\texttt{mxCreateDoubleMatrix} & 创建二维双精度浮点数组\\
\texttt{mxCreateDoubleScale} & 标量,双精度数组(初始化为特定值)\\
\texttt{mxCreateNumericMatrix} & 创建二维数值矩阵\\
\texttt{mxCreateNumericArray} & 创建N维数值数组\\
\texttt{mxCreateUnitintNumericMatrix} & Unintialized 2-D numeric matrix\\
\texttt{mxCreateUninintNumericArray} & 未初始化的N维数组\\
\texttt{mxCreateString} & 一维符号数组,字符串\\
\texttt{mxCreateCharMatrixFromStrings} & 二维字符串数组\\
\texttt{mxStringCharArray} & N维字符串数组\\
\texttt{mxCreateLogicalScalar} & 标量或者逻辑变量数组\\
\texttt{mxCreatelogicalMatrix} & 二维逻辑数组\\
\texttt{mxCreateLogicalArray} & N维逻辑数组\\
\texttt{mxCreateSparseLogicalMatrix} & 二维稀疏逻辑数组\\
\texttt{mxCreateSparse} & 二维稀疏数组\\
\texttt{mxCreateStructMatrix} & 二维 structure数组\\
\texttt{mxCreateStructArray} & N维structure数组\\
\texttt{mxCreateCellMatrix} & 二维cell数组\\
\texttt{mxCreateCellArray} & N维structure数组\\
\texttt{mxDestroyArray} & 释放分配的内存空间\\
\texttt{mxDuplicateArray} & 复制数组空间\\
\texttt{mxCalloc} & 使用matlab的内存管理器分配数组,初始化为0\\
\texttt{mxMalloc} & 使用matlab的内存管理器分配未初始化的数组\\
\texttt{mxRealloc} & 使用matlab的内存管理器再次动态分配内存\\
\texttt{mxFree} & 释放使用matlab内存管理器分配的内存\\
\hline
\end{tabular}
\end{table}
\subsection{验证数据}
\label{sec:orgheadline4}


matlab为验证数据也提供了一些API,这些API的主要作用是判断输入数据的类型。如表\ref{tab:orgtable3}所示。
\begin{table}[htb]
\caption{\label{tab:orgtable3}
验证数据API}
\centering
\begin{tabular}{ll}
\hline
名称 & 功能\\
\hline
\texttt{mxIsDoulbe} & 判断mxArray是不是双精度浮点数\\
\texttt{mxIsSingle} & 判断mxArray是不是单精度浮点数\\
\texttt{mxIsComplex} & 判断是否为复数\\
\texttt{mxIsNumeric} & 判断是否为numeric\\
\texttt{mxIsInt64} & 判断是不是64bit整型\\
\texttt{mxIsUint64} & 判断是不是无符号64bit整型\\
\texttt{mxIsInt32} & 判断是不是32位整型\\
\texttt{mxIsUint32} & 判断是不是无符号32位整型\\
\texttt{mxIsInt16} & 判断是不是16位整型\\
\texttt{mxIsUint16} & 判断是不是无符号16位整型\\
\texttt{mxIsInt8} & 判断是不是16位整型\\
\texttt{mxIsUint8} & 判断是不是无符号16位整型\\
\texttt{mxIsScalar} & 判断是不是scalar数组\\
\texttt{mxIsChar} & 判断是不是符号数组\\
\texttt{mxIsLogical} & 判断是不是逻辑\\
\texttt{mxIsLogicalScalar} & 判断是不是逻辑数组\\
\texttt{mxIsLogicalScalarTrue} & 判断逻辑数组是不是真\\
\texttt{mxIsStruct} & 判断是否为sturcture数组\\
\texttt{mxIsCell} & 判断是不是cell数组\\
\texttt{mxIsClass} & 判断是不是特定的类型\\
\texttt{mxIsInf} & 判断输入是不是无穷大\\
\texttt{mxIsFinite} & 判断输入是不是有限大\\
\texttt{mxIsNaN} & 判断输入是不是NaN\\
\texttt{mxIsEmpty} & 判断数组是否为空\\
\texttt{mxIsSparse} & 判断输入是否为稀疏数组\\
\texttt{mxIsFromGlobalWS} & 判断数组是否从matlab全局空间传入\\
\texttt{mxAssert} & 判断assertion值(用于debug)\\
\texttt{mxAssertS} & 判断assertion值,不打印assertion文本\\
\hline
\end{tabular}
\end{table}

\subsection{读取数据}
\label{sec:orgheadline5}


matlab为数据读写定义了一些API。这些API在 \texttt{mexFunction} 函数中使用非常广泛,极大的方便了matlab和C之间的数据传递。表\ref{tab:orgtable4}给出了这些函数名称和功能简述

\begin{table}[htb]
\caption{\label{tab:orgtable4}
读取数据API}
\centering
\begin{tabular}{ll}
\hline
名称 & 功能简述\\
\hline
\texttt{mxGetNumberOfDimensions} & 获取数组的维度\\
\texttt{mxGetElementSize} & 每一个数据元素需要的字节数\\
\texttt{mxGetDimensions} & 指向数组维度的指针\\
\texttt{mxSetDimensions} & 修改维度和每一维的大小\\
\texttt{mxGetNumberOfElements} & 数组中元素的个数\\
\texttt{mxCalcSingleSubscript} & 相对于起始元素的偏移量\\
\texttt{mxGetM} & 数组元素的行数\\
\texttt{mxGetN} & 数组元素的列数\\
\texttt{mxSetM} & 设置数组元素的行数\\
\texttt{mxSetN} & 设置数组元素的列数\\
\texttt{mxGetScalsr} & 第一个数据元素的实部\\
\texttt{mxGetPr} & 数组的实部\\
\texttt{mxSetPr} & 设置数组的实部\\
\texttt{mxGetPi} & 数组的虚部\\
\texttt{mxSetPi} & 设置数组的虚部\\
\texttt{mxGetData} & 指向数组中实部的指针\\
\texttt{mxSetData} & 设置指向数组中实部的指针\\
\texttt{mxGetImagData} & 指向数组中虚部的指针\\
\texttt{mxSetImagData} & 设置指向数组中虚部的指针\\
\texttt{mxGetChars} & 指向字符数组的指针\\
\texttt{mxGetLogicals} & 指向逻辑类型数组的指针\\
\texttt{mxGetClassID} & 数组的类型\\
\texttt{mxGetClassName} & 数组的类型(以字符串返回)\\
\texttt{mxSetClassName} & 把C中的Array指定为matlab中的array\\
\texttt{mxGetProperty} & matlab对象的公共值\\
\texttt{mxSetProperty} & 设置matlab对象的公共值\\
\texttt{mxGetField} & 获取structure的域(给定index和name)\\
\texttt{mxSetField} & 设置structure的域\\
\texttt{mxGetNumberOfFields} & structure中的域个数\\
\texttt{mxGetFieldNameByNumber} & 给定域编号获得域的名字\\
\texttt{mxGetFieldNumber} & 给定域名字获得域编号\\
\texttt{mxGetFieldByNumber} & 给定域索引和域数值获得域值\\
\texttt{mxSetFieldByNumber} & 给定索引和域数值,设置域值\\
\texttt{mxAddField} & 在结构体中添加域\\
\texttt{mxRemoveField} & 在结构体中去除域\\
\texttt{mxGetCell} & 获取Cell数组的值\\
\texttt{mxSetCell} & 设置Cell数组的值\\
\texttt{mxGetNzmax} & IR,PR,PI数组的元素个数\\
\texttt{mxSetNzmax} & 为非零元素设置空间数\\
\texttt{mxGetIr} & 稀疏IR数组\\
\texttt{mxSetIr} & 稀疏数组的IR数组\\
\texttt{mxGetJc} & 稀疏JC数组\\
\texttt{mxSetJc} & JC数组的稀疏数组\\
\hline
\end{tabular}
\end{table}

\subsection{数据类型转换}
\label{sec:orgheadline6}


matlab为字符和数组之间的数据类型转换提供了API,如表tab:20151111ConvertDataTypes所示。
\begin{table}[htb]
\caption{\label{tab:orgtable5}
数据类型转换}
\centering
\begin{tabular}{ll}
\hline
名称 & 功能简述\\
\hline
\texttt{mxArrayToString} & array向字符串转化\\
\texttt{mxArrayToUTF8String} & array向字符串转化(用UTF8编码)\\
\texttt{mxGetString} & 把string数组转化为C-style字符串\\
\texttt{mxSetClassName} & 结构体数组转向matlab对象数组\\
\hline
\end{tabular}
\end{table}
\end{document}
