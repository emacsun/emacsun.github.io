\documentclass[10pt,a4paper,UTF8]{article}
\usepackage{zclorg}
\author{zcl.space}
\date{}
\title{Matlab与C混合编程API之mxCreateNumericMatrix}
\hypersetup{
 pdfauthor={zcl.space},
 pdftitle={Matlab与C混合编程API之mxCreateNumericMatrix},
 pdfkeywords={matlab communication simulation C},
 pdfsubject={mxCreateNumericMatrix创建二维数组,可以指定类型},
 pdfcreator={Emacs 25.0.50.1 (Org mode 8.3.2)}, 
 pdflang={English}}
\begin{document}

\maketitle\xiaosihao
\tableofcontents\newpage\newpage


\section{引言}
\label{sec:orgheadline1}


在\href{matlabmxCreateDoulbeMatrix.org}{Matlab与C混合编程API之mxCreateDoubleMatrix} 中,我们使用 \texttt{mxCreateDoubleMatrix} 创建二维双精度矩阵。本文要介绍的 \texttt{mxCreateNumericMatrix} 比  \texttt{mxCreateDoubleMatrix} 用途更广, \texttt{mxCreateNumericMatrix} 也用于创建二维矩阵,但是其可以指定矩阵元素的类型,包括整型和浮点类型。
\section{使用语法}
\label{sec:orgheadline2}


\lstset{language=C,label= ,caption= ,captionpos=b,numbers=none}
\begin{lstlisting}
#include "mex.h"
mxArray *mxCreateNumericMatrix(mwSize m, mwSize n,
  mxClassID classid, mxComplexity ComplexFlag);
\end{lstlisting}

输入参数 \texttt{m} \texttt{n} 和 \texttt{ComplexFlag} 就不做过多的介绍,请见\href{matlabmxCreateDoulbeMatrix.org}{Matlab与C混合编程API之mxCreateDoubleMatrix} 一文。这里着重介绍一下 \texttt{mxClassID} 。这个参数表示了矩阵中元素的类型,matlab根据这个类型解释内存中二进制比特的值。比如在C中设置这个值为 \texttt{mxINT16\_CLASS} 表示矩阵元素都是16位整型。桥梁函数中的类型与matlab中类型对照表如下:
\begin{table}[htb]
\caption{\label{tab:orgtable1}
matlab和C类型对照表}
\centering
\begin{tabular}{ll}
\hline
matlab类型 & 桥梁函数中对应类型\\
\hline
int8 & mxINT8\_CLASS\\
uint8 & mxUINT8\_CLASS\\
int16 & mxINT16\_CLASS\\
uint16 & mxUINT16\_CLASS\\
int32 & mxINT32\_CLASS\\
uint32 & mxUINT32\_CLASS\\
int64 & mxINT64\_CLASS\\
uint64 & mxUINT64\_CLASS\\
single & mxSINGLE\_CLASS\\
double & mxDOUBLE\_CLASS\\
\hline
\end{tabular}
\end{table}
\section{一个例子}
\label{sec:orgheadline3}


在学习新东西的过程中,我比较喜欢例子。在和别人交流新概念的时候,我也比较喜欢使用例子。接下来,用一个小例子来阐述 \texttt{mxCreateNumericMatrix} 的使用。这个例子使用C创建一个矩阵被matlab使用。代码如下:
\lstset{language=C,label= ,caption= ,captionpos=b,numbers=left}
\begin{lstlisting}
#include "mex.h"

/* The mxArray in this example is 2x2 */
#define ROWS 2
#define COLUMNS 2
#define ELEMENTS 4

void mexFunction(int nlhs, mxArray *plhs[], int nrhs,
		 const mxArray *prhs[])
    /* pointer to real data in new array */
    double  *pointer;
    mwSize index;
    /* existing data */
    const double data[] = {2.1, 3.4, 2.3, 2.45};

    /* Check for proper number of arguments. */
    if ( nrhs != 0 ) {
	mexErrMsgIdAndTxt("MATLAB:arrayFillGetPr:rhs",
	    "This function takes no input arguments.");
    }

    /* Create an m-by-n mxArray; you will copy
       existing data into it */
    plhs[0] = mxCreateNumericMatrix(ROWS, COLUMNS,
			   mxDOUBLE_CLASS, mxREAL);
    pointer = mxGetPr(plhs[0]);

    /* Copy data into the mxArray */
    for ( index = 0; index < ELEMENTS; index++ ) {
	pointer[index] = data[index];
    }
    return;
}
\end{lstlisting}

由于功能简单,在桥梁函数中就没有重新调用函数。在这个例子中生成一个 \(2\times 2\)的矩阵,其中的元素为:[2.1 2.3;3.4 2.45],注意这里的元素顺序,在matlab中元素的位置是按列存放的。所以matlab中的矩阵[2.1 2.3;3.4 2.45],表示成一维数组就是[2.1 3.4 2.3 2.45]。

代码的第24行调用了 \texttt{mxCreateNumericMatrix} ,并指定了 \texttt{mxClassID}  为 \texttt{mxDOUBLE\_CLASS} 。注意:代码的第34行的 \texttt{pointer} 不能用 \texttt{plhs[0]} 代替,即不能写成:
\lstset{language=C,label= ,caption= ,captionpos=b,numbers=none}
\begin{lstlisting}
plhs[0][index] = data[index]
\end{lstlisting}
这是因为,在桥梁函数中 \texttt{plhs[0]} 是被matlab代码使用的地址,其指向 \texttt{mxArray} 类型变量。而  \texttt{pointer} 是被C使用的地址,其指向 \texttt{double} 类型变量。
\section{创建矩阵的其它几个API}
\label{sec:orgheadline7}


本想把matlab里创建数组的API一个一片博文写出来,后来发现这些API大同小异。如果我还坚持初衷,未免显得累赘,有凑数之嫌(我是那种靠数量取胜的人么?)。
\subsection{mxCreateUninitNumericMatrix}
\label{sec:orgheadline4}


与 \texttt{mxCreateNumericMatrix} 相比,这个API的唯一区别是创建的矩阵没有初始化,matlab你能告诉我为什么还要定义这样一个API么?我怎么发现matlab有凑数之嫌呢?用一个能说服我的理由拍醒我吧!
\subsection{mxCreateNumericArray}
\label{sec:orgheadline5}


与 \texttt{mxCreateNumericMatrix} 相比,这个API的区别在于创建的矩阵不限于二维,可以是N维,所以其调用语法略有不同,如下:
\lstset{language=C,label= ,caption= ,captionpos=b,numbers=none}
\begin{lstlisting}
#include "mex.h"
mxArray *mxCreateNumericArray(mwSize ndim, const mwSize *dims,
	 mxClassID classid, mxComplexity ComplexFlag);
\end{lstlisting}
其中 \texttt{ndim} 指定了要创建的矩阵维度, \texttt{dims} 是指向表示维度的数组的指针, \texttt{dim[0]} 表示第一维的大小,依次类推。该API创建的N维矩阵所有元素都被初始化为0.
\subsection{mxCreateUninitNumericArray}
\label{sec:orgheadline6}


从名字上就可以看出来和 \texttt{mxCreateNumericArray} 的区别。不说了,matlab你定义这个API就是在耍流氓。

\section{尾声}
\label{sec:orgheadline8}



本文介绍了 \texttt{mxCreateNumericMatrix} API的语法和使用过程,并指出 \texttt{plhs[0]} 的一个使用限制。
\end{document}
