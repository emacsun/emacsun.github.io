\documentclass[10pt,a4paper,UTF8]{article}
\usepackage{zclorg}
\author{emacsun}
\date{}
\title{Thinking in Java chapter4 笔记和习题}
\hypersetup{
 pdfauthor={emacsun},
 pdftitle={Thinking in Java chapter4 笔记和习题},
 pdfkeywords={},
 pdfsubject={},
 pdfcreator={Emacs 25.0.50.1 (Org mode 8.3.4)}, 
 pdflang={English}}
\begin{document}

\maketitle
\tableofcontents

本章学习起来毫无压力,因为大多和 \texttt{C/C++} 一脉相承。只记录几个和 \texttt{C/C++} 不一样的点,以备复习时多加注意。

\section{Java赋值}
\label{sec:orgheadline1}


第四章给出带有一个整数成员的类的赋值,
\lstset{language=java,label= ,caption= ,captionpos=b,firstnumber=1,numbers=left}
\begin{lstlisting}
//assignment with objects is a bit tricky
import static net.mindview.util.Print.*;

class Tank{
    int level;
}

public class assignment{
    public static void main(String[] args) {
        Tank t1 = new Tank();
        Tank t2 = new Tank();
        t1.level = 47;
        t1.level = 27;
        print("1:t1.level: " + t1.level +
              ", t2.level: " + t2.level);
        t1 = t2;
        print("2:t1.level: " + t1.level +
              ", t2.level:" + t2.level);
        t1.level = 33;
        print("3:t1.level: " + t1.level +
              ", t2.level:" + t2.level);
        t2.level = 99;
        print("4:t1.level: " + t1.level +
              ", t2.level:" + t2.level);
    }
}
\end{lstlisting}
在第16行执行结束之后, \texttt{t1} 和 \texttt{t2} 变成同一个对象,原先 \texttt{t2} 指向的对象会被garbage collector收走。如果不想 \texttt{t2} 被收走,则需要对对象的每个域进行赋值操作:

\lstset{language=java,label= ,caption= ,captionpos=b,numbers=none}
\begin{lstlisting}
t1.level = t2.level;
\end{lstlisting}


接下来给出对应的 \texttt{float} 类型,也就是第四章练习2.

\lstset{language=java,label= ,caption= ,captionpos=b,firstnumber=1,numbers=left}
\begin{lstlisting}
//assignment with objects is a bit tricky
import static net.mindview.util.Print.*;

class Tank{
    float level;
}

public class assignment{
    public static void main(String[] args) {
        Tank t1 = new Tank();
        Tank t2 = new Tank();
        t1.level = 47.47f;
        t1.level = 27.27f;
        print("1:t1.level: " + t1.level +
              ", t2.level: " + t2.level);
        t1 = t2;
        print("2:t1.level: " + t1.level +
              ", t2.level:" + t2.level);
        t1.level = 33.33f;
        print("3:t1.level: " + t1.level +
              ", t2.level:" + t2.level);
        t2.level = 99.99f;
        print("4:t1.level: " + t1.level +
              ", t2.level:" + t2.level);
    }
}
\end{lstlisting}

注意在赋值程序对应的 \texttt{float} 版本中,浮点数的赋值 \texttt{t1.level = 47.47f} 这个 \texttt{47.47f} 如果改成 \texttt{47.47} 在我的编译器 \texttt{JSE8} 上会出错。一定是有一个开关,可以允许这样的赋值,这个开关就是 \texttt{cast} 即类型转换。

Java 是强类型语言,默认的 \texttt{int} 变量初始化值是 \texttt{0} ; \texttt{long} 变量初始化值是 \texttt{0L} ; \texttt{float} 变量初始化值是 \texttt{0.0f} ; \texttt{double} 初始化值是 \texttt{0.0d}.

当向一个函数传递对象的时候,也要注意传递的是引用,如下:
\lstset{language=java,label= ,caption= ,captionpos=b,firstnumber=1,numbers=left}
\begin{lstlisting}
//assignment with objects is a bit tricky
import static net.mindview.util.Print.*;

class Letter{
    char c;
    float ff;
}

public class PassObject{
    static void f(Letter y){
        y.c = 'z';
        y.ff = 12.3445f;
    }
    public static void main(String[] args) {
        Letter x = new Letter();
        x.c = 'a';
        x.ff = 234.45f;
        print("1:x.c " + x.c +
              "x.ff " + x.ff);
        f(x);
        print("2:x.c " + x.c +
              "x.ff " + x.ff);
    }
}
\end{lstlisting}

输出是:
\begin{verbatim}
1:x.c ax.ff 234.45
2:x.c zx.ff 12.3445
\end{verbatim}
可以看到当把 \texttt{x} 传递给 \texttt{f()} 时,传递给 \texttt{f()} 的是引用,在 \texttt{f()} 中执行的操作直接影响到 \texttt{x} 的值,这与 \texttt{C/C++} 中不同。
\section{操作符}
\label{sec:orgheadline2}



\lstset{language=java,label= ,caption= ,captionpos=b,firstnumber=1,numbers=left}
\begin{lstlisting}
import java.util.*;
import static net.mindview.util.Print.*;

public class MathOps
{
    public static void main(String args[])
    {
        //create a seeded random number generator
        Random rand = new Random(47);
        int i,j,k;
        j = rand.nextInt(100) +1;
        k = rand.nextInt(100) +1;
        i = j + k;
        print("j ( " + j +
              " ) + k ( " + k +
              " ) =  " + i);
    }
}
\end{lstlisting}
在这段程序中,为了生成一个随机数,调用了 \texttt{Random} 这个 \texttt{Class} 。当没有指定随机数种子时,随机数种子是当前时间。

除了 \texttt{nextInt()} 这个方法外,还有其他方法,比如 \texttt{nextfloat(), nextdouble()}  详见这个类的方法总结。

\lstset{language=java,label= ,caption= ,captionpos=b,firstnumber=1,numbers=left}
\begin{lstlisting}
import java.util.*;
import static net.mindview.util.Print.*;

public class MathOps
{
    public static void main(String args[])
    {
        //create a seeded random number generator
        Random rand = new Random(47);
        int i,j,k;
        j = rand.nextInt(100) +1;
        k = rand.nextInt(100) +1;
        i = j + k;
        print("j ( " + j +
              " ) + k ( " + k +
              " ) =  " + i);
        float i,j,k; 
        j = rand.nextFloat(100) +1;
        k = rand.nextFloat(100) +1;
        i = j + k;
        print("j ( " + j +
              " ) + k ( " + k +
              " ) =  " + i);

        double i,j,k; 
        j = rand.nextDouble(100) +1;
        k = rand.nextDouble(100) +1;
        i = j + k;
        print("j ( " + j +
              " ) + k ( " + k +
              " ) =  " + i);
    }
}
\end{lstlisting}
上述代码会报错:错误信息是, \texttt{i,j,k} 已经定义了,这在 \texttt{C/C++} 中是允许的,但是在Java中却是不允许的。

看如下代码:
\lstset{language=java,label= ,caption= ,captionpos=b,firstnumber=1,numbers=left}
\begin{lstlisting}
import java.util.*;
import static net.mindview.util.Print.*;
public class ExerciseCh0404
{
    public static void main(String args[])
    {
        int distance=100;
        int time=300;
        float velocity;
        velocity = (float) distance / time;
        print("distance / time = " + velocity);
    }
}
\end{lstlisting}
如果在第10行不加 \texttt{float} 显示的指示一下,结果输出就是:
\begin{verbatim}
distance / time = 0.00
\end{verbatim}
如果加 \texttt{float} 指示转换除法结果为的类型就会把计算结果转换为浮点类型。
\begin{verbatim}
distance / time = 0.333333334
\end{verbatim}
\section{关系运算符}
\label{sec:orgheadline3}

在Java中一切都是对象。所以针对对象的比较会带来一些意想不到的结果。看如下代码:
\lstset{language=java,label= ,caption= ,captionpos=b,firstnumber=1,numbers=left}
\begin{lstlisting}
public class Equivalence
{
    public static void main(String args[])
    {
        Integer n1 = new Integer(47);
        Integer n2 = new Integer(47);
        System.out.println(n1 == n2);
        System.out.println(n1 != n2);
    }
}
\end{lstlisting}
输出结果是:
\begin{verbatim}
false
true
\end{verbatim}
确实surprise! 两个内容相同的整数(都是47),但我们判断它们是否相等时,结果居然是 \texttt{false} 。这是因为在Java中== 这个关系运算符比较的不仅仅是内容,还比较 \texttt{reference} .  想要比较内容是否相等正确的做法是:使用方法 \texttt{equals()}
\lstset{language=java,label= ,caption= ,captionpos=b,firstnumber=1,numbers=left}
\begin{lstlisting}
public class Equivalence
{
    public static void main(String args[])
    {
        Integer n1 = new Integer(47);
        Integer n2 = new Integer(47);
        System.out.println(n1.equals(n2));
    }
}
\end{lstlisting}
输出结果是:
\begin{verbatim}
true
\end{verbatim}
方法 \texttt{equals()} 是 \texttt{Integer} 内置的一个方法,用来比较内容是否相等。同理, \texttt{Float} 类的对象也有 \texttt{equals()} 方法。

\lstset{language=java,label= ,caption= ,captionpos=b,firstnumber=1,numbers=left}
\begin{lstlisting}
public class Equivalence
{
    public static void main(String args[])
    {
        Float n1 = new Float(47.0456f);
        Float n2 = new Float(47.0456f);
        System.out.println(n1.equals(n2));
    }
}
\end{lstlisting}
输出结果是:
\begin{verbatim}
true
\end{verbatim}
这还没完,见如下代码:
\lstset{language=java,label= ,caption= ,captionpos=b,firstnumber=1,numbers=left}
\begin{lstlisting}
class Value {
    int i;
}

public class Equivalence
{
    public static void main(String args[])
    {
        Value n1 = new Value();
        Value n2 = new Value();
        n1.i = n2.i = 100;
        System.out.println(n1.equals(n2));
    }
}
\end{lstlisting}
你认为结果是什么?会是 \texttt{true} 么?如果你认为是 \texttt{true} ,那么你就错了。结果是 \texttt{false} 。这是因为 \texttt{equals()} 函数默认比较的是 \texttt{reference} 。如果要 \texttt{equals()} 函数比较自定义对象的内容,那么你得重载 \texttt{equals()} 这个函数。那么,如果我照如下方法写代码呢?
\lstset{language=java,label= ,caption= ,captionpos=b,firstnumber=1,numbers=left}
\begin{lstlisting}
class Value {
    int i;
}

public class Equivalence
{
    public static void main(String args[])
    {
        Value n1 = new Value();
        Value n2 = new Value();
        n1.i = n2.i = 100;
        System.out.println(n1.i.equals(n2.i));
    }
}
\end{lstlisting}
编译报错:
\begin{verbatim}
Equivalence.java:12: error: int cannot be dereferenced
        System.out.println(n1.i.equals(n2.i));
\end{verbatim}
\section{逻辑运算符}
\label{sec:orgheadline4}


Java语言中对逻辑云算法的操作也不同于 \texttt{C/C++}. 在 \texttt{C/C++} 中,非零值都是 \texttt{true} ,只有零会被当成 \texttt{false} . 但是在Java中却不是:在Java中,不能把非零值或者零当做 \texttt{boolean} 来处理, \texttt{boolean} 值只有两种 \texttt{true} 或者 \texttt{false} . 看代码:
\lstset{language=java,label= ,caption= ,captionpos=b,firstnumber=1,numbers=left}
\begin{lstlisting}
import java.util.*;
import static net.mindview.util.Print.*;

public class Bool
{
    public static void main(String args[])
    {
        Random rand = new Random(47);
        int i = rand.nextInt(100);
        int j = rand.nextInt(100);
        print("i = " + i );
        print("j = " + j );
        print("i < j is " + (i<j));
        print("i > j is " + (i>j));
        print("i <= j is " + (i<=j));
        print("i >= j is " + (i>=j));
        print("i == j is " + (i==j));
        print("i != j is " + (i!=j));
        print("i && j is " + (i&&j));
    }
}
\end{lstlisting}
这个代码的第19行会报错,因为 \texttt{\&\&} 这样的逻辑运算符只针对 \texttt{boolean} 类型,在 \texttt{C/C++} 中这样的操作是允许的。但是在 \texttt{Java} 中却不是允许的。所以 \texttt{Java}  是比 \texttt{C/C++} 更强的有类型语言,在写代码的时候最好把类型转化这样的写清楚,不要指望系统自动去转换,这样的代码也更好懂。
\section{Java中的数值表示}
\label{sec:orgheadline5}


前面说过 \texttt{Java} 是一种强类型语言,甚至比 \texttt{C/C++} 还要强类型。因此在数值表示的时候要格外小心,否则在编译阶段就不会通过。这对于写出健壮的代码是非常有用的。看代码:
\lstset{language=java,label= ,caption= ,captionpos=b,firstnumber=1,numbers=left}
\begin{lstlisting}
import static net.mindview.util.Print.*;

public class ExerciseCh0408
{
    public static void main(String args[])
    {
        byte i1 = 0X80;
        print("i1 = " + Integer.toBinaryString(i1));
        print("i1 = " + Long.toBinaryString(i1));
    }
}
\end{lstlisting}
上面的代码对一个 \texttt{byte} 类型的变量赋值 \texttt{0x80} 这在边一阶段就会报错:
\begin{verbatim}
error: incompatible types: possible lossy conversion from int to byte
        byte i1 = 0X80;
\end{verbatim}
提醒我,从 \texttt{int} 到 \texttt{byte} 的类型转换会出错。因为 \texttt{0x80} 是 \texttt{int} 范围内的数, \texttt{byte} 表示不了这个数值。 \texttt{byte} 只能表示 \texttt{-0x80} 。

如果把 \texttt{0x80} 改成 \texttt{-0x80} 则会看到输出:
\begin{verbatim}
i1 = 11111111111111111111111110000000
i1 = 1111111111111111111111111111111111111111111111111111111110000000
\end{verbatim}

注意方法 \texttt{Integer.toBinaryString} 把 \texttt{byte} 变换成 \texttt{int} 来处理, \texttt{Long.toBinaryString} 把 \texttt{byte} 变换成 \texttt{long} 来处理而 \texttt{Byte} 类却没有方法 \texttt{toBinaryString()} 不知道 \texttt{Java} 是怎么想的。 \texttt{-0x80} 表示的是 \texttt{-128} 这个数是 \texttt{Byte} 能表示的最小的数。在Java中所有的数都是有符号数。
\section{指数表示}
\label{sec:orgheadline6}


在脚本语言 \texttt{MATLAB} 以及 \texttt{C/C++} 中 \texttt{aEb} 表示 \(a\times 10^{b}\)。在 \texttt{Java} 中,也有这种指数表示,不过由于 \texttt{Java} 是强类型语言,所以还是稍有不同。看代码:
\lstset{language=java,label= ,caption= ,captionpos=b,firstnumber=1,numbers=left}
\begin{lstlisting}
public class Exponents
{
    public static void main(String args[])
    {
        float expfloat = 1.25e-43f;
        float expFloat = 1.25E-43f;
        System.out.println(expfloat);
        System.out.println(expFloat);
        double expdouble = 47e47d;
        double expDouble = 47e47;
        System.out.println(expDouble);
    }
}
\end{lstlisting}
从这段代码中可以知道: \texttt{e} 和 \texttt{E} 是通用的。还有一些隐藏的信息: \texttt{aEb} 在 \texttt{Java} 中默认是 \texttt{Double} 类型。所以代码第9行的 \texttt{47e47d} 的 \texttt{d} 是可选的,第10行就去掉了。进而第5行和第6行的 \texttt{f} 是必须的,这里的 \texttt{1.25e-43f} 相当于 \texttt{(float) 1.25e-43} 或者 \texttt{(float) 1.25e-43D} .
\section{位操作}
\label{sec:orgheadline7}


虽然没有 \texttt{C/C++} 中位操作灵活,但是Java中也有对应的位操作,看代码:
\lstset{language=java,label= ,caption= ,captionpos=b,firstnumber=1,numbers=left}
\begin{lstlisting}
import static net.mindview.util.Print.*;
public class ExerciseCh0410
{
    public static void main(String args[])
    {
        int a = 0xBBBB;
        int b = 0x5555;
        print("a = " + Integer.toBinaryString(a));
        print("b = " + Integer.toBinaryString(b));
        print("a &= b " + Integer.toBinaryString(a&=b));
        print("a ^= b " + Integer.toBinaryString(a^=b));
        print("a |= b " + Integer.toBinaryString(a|=b));
    }
}
\end{lstlisting}
输出为:
\begin{verbatim}
a = 1011101110111011
b = 101010101010101
a &= b 1000100010001
a ^= b 100010001000100
a |= b 101010101010101
\end{verbatim}
\end{document}
