% Intended LaTeX compiler: pdflatex
\documentclass[10pt,a4paper,UTF8]{article}
\usepackage{zclorg}
\author{emacsun}
\date{}
\title{BRML第一章:概率推断}
\hypersetup{
 pdfauthor={emacsun},
 pdftitle={BRML第一章:概率推断},
 pdfkeywords={},
 pdfsubject={},
 pdfcreator={Emacs 25.0.50.1 (Org mode 9.0.5)}, 
 pdflang={English}}
\begin{document}

\maketitle
\tableofcontents

最近我在阅读 David Barber 的 Bayesian Reasoning and Machine Learning(以后都用BRML来简称)。阅读过程中,随手记录了一些笔记。现在稍作整理,发表为一系列博文。本文是第一篇,也就是第一章的笔记整理。后续的博文都是各个章节的笔记整理,基本是一章一篇博文,个别重要的章节或许会作为多篇博文来写。

BRML的第一章主要复习了概率的基本知识,通过贝叶斯公式(或者条件概率公式)引入概率推断的概念,然后对先验概率,后验概率和似然值做了详细介绍。从第一章就可以窥测本文将会用大量的例子来阐述概念,这是我比较喜欢的书籍的风格。


\section{贝叶斯法则}
\label{sec:org22c22ff}


为了保证完整性,记录贝叶斯公式的定义。

\begin{definition}
已知事件 \(y\)时,事件\(x\)发生的概率,定义为:\[p(x|y) \triangleq \frac{p(x,y)}{p(y)} = \frac{p(x)p(y|x)}{p(y)}\]
\end{definition}
这个公式是如此重要,基本上撑起了机器学习的半壁江山。另外,这个公式在通信系统的信道译码算法中也频频出现,尤其在消息传递算法或者置信度传递算法中。关于其重要性,就不再多言。随着学习的深入,对这个公式及其扩展的理解会愈加深刻。

从贝叶斯规则引入统计独立的概念,即当\[p(x,y) = p(x)p(y)\] \(x,y\)是相互独立的。此时\(x\)的发生不影响\(y\)发生的概率,即\[p(y|x) = p(y)\]

\section{概率推断}
\label{sec:orgdd14d83}


概率推断的核心是识别环境中所有相关的随机变量\(x_{1},\ldots ,x_{N}\),并且根据他们的关系创建一个概率模型\(p(x_{1},\ldots ,x_{N})\)。推断的过程是根据已知信息更新某个随机变量概率的过程。

\begin{instance}
医生发现一个人得KJ病的概率是相当低的大约是\(1/100000\),但是得了Kreuzfeld-Jacob(KJ)病的人几乎都吃汉堡包,\(p(Hamburger Eater| KJ) = 0.9\)。
\begin{enumerate}
\item 假设一个人吃汉堡包的概率是0.5,\(p(Hamburtger Eater) = 0.5\),那么一个吃汉堡包的人得KJ的概率是多少?

这个概率可以表示为:
\begin{eqnarray}
\label{eq:1}
p(KJ| Hamburtger Eater) &=& \frac{p(KJ,Hamburtger Eater)}{p(Hamburtger Eater)} \\
 &=& \frac{p(Hamburtger Eater | KJ)p(KJ)}{p(Hamburtger Eater)} \\
&=& \frac{0.9\times 1/100000}{1/2} \\
&=& 1.8\times 10^{-5}
\end{eqnarray}

\item 如果吃汉堡包的人的概率比较低,不是0.5而是0.001,那么一个吃汉堡包的人得KJ的概率是多少?

重复上面的计算
\begin{eqnarray}
\label{eq:2}
p(KJ| Hamburtger Eater) &=& \frac{p(KJ,Hamburtger Eater)}{p(Hamburtger Eater)}\\ 
&=& \frac{p(Hamburtger Eater | KJ)p(KJ)}{p(Hamburtger Eater)} \\
&=& \frac{0.9\times 1/100000}{0.001} \\
&\approx&  1/100
\end{eqnarray}
\end{enumerate}
\end{instance}
这个例子告诉我们不要为一些不大可能的事情担心:得了KJ的前提下吃汉堡包的概率和吃汉堡包的前提下得KJ的概率完全是两码事。


\begin{instance}
再给一个异或门的例子,我们知道一个标准的异或门电路的逻辑关系满足表\ref{tab:org226f04a}:
\begin{table}[htbp]
\caption{\label{tab:org226f04a}
异或门逻辑}
\centering
\begin{tabular}{center}
\hline
\(A\) & \(B\) & \(C=A\oplus B\)\\
\hline
0 & 0 & 0\\
0 & 1 & 1\\
1 & 0 & 1\\
1 & 1 & 1\\
\hline
\end{tabular}
\end{table}

当我们观测到异或门的输出是\(0\)时,对于\(A\)或者\(B\)的概率我们知道多少?在这样的情况下,可能\(A,B\)都是\(0\),也可能\(A,B\)都是\(1\)。 \(A,B\)处于\(0\)或者\(1\)的概率是等该的。

但是考虑一个软判决输出的异或门逻辑,其逻辑关系如表\ref{tab:orge6ef604},我们假定\(A,B\)是独立的且\(p(A=1)=0.65,p(B=1)=0.77\),那么求\(p(A=1|C=0)\)?
\begin{table}[htbp]
\caption{\label{tab:orge6ef604}
异或门逻辑}
\centering
\begin{tabular}{center}
\hline
\(A\) & \(B\) & \(p(C=1  \vert  A,B)\)\\
\hline
0 & 0 & 0.1\\
0 & 1 & 0.99\\
1 & 0 & 0.8\\
1 & 1 & 0.25\\
\hline
\end{tabular}
\end{table}

由条件概率公式,得:
\begin{equation}
\label{eq:3}
p(A=1|C=0) = \frac{p(A=1,C=0)}{p(C=0)} = \frac{p(A=1,C=0)}{p(A=1,C=0) + p(A=0,C=0)}
\end{equation}
接下来,我们对\ref{eq:3}右边分母上的两个求和项进行展开:
\begin{eqnarray}
\label{eq:4}
p(A=1,C=0) &=&\sum_{B}p(A=1,B,C=0) \\
           &=& \sum_{B}p(C=0|B,A=1)p(B)p(A=1)
\end{eqnarray}
\begin{eqnarray}
\label{eq:5}
p(A=0,C=0) &=&\sum_{B}p(A=0,B,C=0) \\ 
&=& \sum_{B}p(C=0|B,A=0)p(B)p(A=0)
\end{eqnarray}
带入表格中相应的概率数字得出\(p(A=1|C=0) = 0.8436\)
\end{instance}
\section{先验概率,似然值和后验概率}
\label{sec:org37a9fb1}


现实生活中非常多的问题可以归类为:当我知道数据\(D\)时,告诉我随机变量\(\theta\)的概率。这个问题可以归类为:
\begin{equation}
\label{eq:6}
p(\theta |D) = \frac{p(D|\theta)p(\theta)}{p(D)} = \frac{p(D|\theta)p(\theta)}{\int_{\theta}p(D|\theta)p(\theta)} 
\end{equation}
这个模型在机器学习和信道编码理论中都有普遍的实用,甚至是其基础的基础。那么从式\ref{eq:6}我们可以读出什么信息呢?式\ref{eq:6}告诉我们,我们可以从数据生成模型\(p(D|\theta)\)和先验概率\(p(\theta)\)推断后验概率\(p(\theta |D)\)。最大后验概率准则(maximize a posteriori,MAP )准则,可以表示为:
\begin{equation}
\label{eq:7}
\hat{\theta} = \arg \max_{\theta} p(\theta | D)
\end{equation}

对于等概率分布的先验概率\(p(\theta)\),MAP准则和最大似然准则(maximum likelihood,ML) 是等效的,即最大化\(p(D|\theta)\)的\(\theta\)同样最大化\(p(\theta|D)\)。

\begin{instance}
现在针对\(p(D|\theta)\)我们给出一个例子。假设一个钟摆在摆动,我们用\(x_{t}\)来表示钟摆在\(t\)时刻的角度。假设每次测量都是独立的。假设每次测量都是精确的,则有
\begin{equation}
\label{eq:8}
x_{t} = \sin(\theta t)
\end{equation}
这里假设系统没有阻尼则\(\theta = \sqrt{g/L}\),其中\(g\)是地球引力场数,\(L\)是吊起钟摆的绳子的长度,但是我们是在测试\(\theta\),是根据\(x_{1},\ldots ,x_{T}\)来测量\(\theta\)。另外,实际测量过程中(比如测量位置仪器质量很差或者定时器不准),测量总是存在误差,假设误差是\(\epsilon_{t}\),测量结果可以表示为:
\begin{equation}
\label{eq:9}
x_{t} = \sin(\theta t) + \theta_{t}
\end{equation}
一般情况我们定义\(\epsilon_{t}\)服从均值为零方差为\(\delta^{2}\)的高斯随机变量。所以关于\(\theta\)的后验概率可以表示为:
\begin{equation}
\label{eq:10}
p(\theta|x_{1},\ldots,x_{T}) \propto p(\theta)\prod_{t=1}^{T}\frac{1}{\sqrt{2\pi \delta^{2}}}e^{\frac{1}{2\delta^{2}}(x_{t}-\sin(\theta)t)^{2}}
\end{equation}
\end{instance}
\begin{instance}
我们来考虑投掷两个均匀骰子的场景。假设现在有人告诉你两个骰子的数字之和为\(9\)。求此时关于两个骰子上数字的后验概率分布。

首先我们用\(s_{a},s_{b}\)代表两个骰子的数字,其取值范围是\(\{1,2,3,4,5,6\}\)。两者之和为\(t=s_{a} + s_{b}\)这三个随机变量的模型遵循:
\begin{equation}
\label{eq:11}
p(t,s_{a},s_{b}) = \underbrace{p(t|s_{a},s_{b})}_{likelihood}\underbrace{p(s_{a},s_{b})}_{prior}
\end{equation}
假设两个骰子是均匀的\(p(s_{a},s_{b})\)则\(p(s_{a},s_{b}) = p(s_{a})p(s_{b})\)其概率分布表格为:

\begin{table}[htbp]
\caption{\label{tab:orgd5cd133}
\(p(s_{a})p(s_{b})\)}
\centering
\begin{tabular}{lllllll}
\hline
 & \(s_{a} = 1\) & \(s_{a} = 2\) & \(s_{a} = 3\) & \(s_{a} = 4\) & \(s_{a} = 5\) & \(s_{a} = 6\)\\
\hline
\(s_{a} = 1\) & 1/36 & 1/36 & 1/36 & 1/36 & 1/36 & 1/36\\
\(s_{a} = 2\) & 1/36 & 1/36 & 1/36 & 1/36 & 1/36 & 1/36\\
\(s_{a} = 3\) & 1/36 & 1/36 & 1/36 & 1/36 & 1/36 & 1/36\\
\(s_{a} = 4\) & 1/36 & 1/36 & 1/36 & 1/36 & 1/36 & 1/36\\
\(s_{a} = 5\) & 1/36 & 1/36 & 1/36 & 1/36 & 1/36 & 1/36\\
\(s_{a} = 6\) & 1/36 & 1/36 & 1/36 & 1/36 & 1/36 & 1/36\\
\hline
\end{tabular}
\end{table}

由于骰子是均匀的则\(p(s_{a}) = p(s_{b}) = 1/6\)。另外,我们有\(p(t|s_{a},s_{b})\),表格如下:
\begin{table}[htbp]
\caption{\label{tab:org8562b97}
\(p(t|s_{a},s_{b})\)}
\centering
\begin{tabular}{lrrrrrr}
\hline
 & \(s_{a} = 1\) & \(s_{a} = 2\) & \(s_{a} = 3\) & \(s_{a} = 4\) & \(s_{a} = 5\) & \(s_{a} = 6\)\\
\hline
\(s_{a} = 1\) & 0 & 0 & 0 & 0 & 0 & 0\\
\(s_{a} = 2\) & 0 & 0 & 0 & 0 & 0 & 0\\
\(s_{a} = 3\) & 0 & 0 & 0 & 0 & 0 & 1\\
\(s_{a} = 4\) & 0 & 0 & 0 & 0 & 1 & 0\\
\(s_{a} = 5\) & 0 & 0 & 0 & 1 & 0 & 0\\
\(s_{a} = 6\) & 0 & 0 & 1 & 0 & 0 & 0\\
\hline
\end{tabular}
\end{table}

后验概率 \(p(s_{a},s_{b}|t=9) = \frac{p(t=9|s_{a},s_{b})p(s_{a})p(s_{b})}{p(t=9)}\),其中\[p(t=9) = \sum_{s_{a}s_{b}} p(t=9| s_{a},s_{b})p(s_{a})p(s_{b})\] 

综上我们可以得到后验概率表:

\begin{table}[htbp]
\caption{\label{tab:orgf8e976b}
\(p(s_{a},s_{b}|t=9)\)}
\centering
\begin{tabular}{lrrrrrr}
\hline
 & \(s_{a} = 1\) & \(s_{a} = 2\) & \(s_{a} = 3\) & \(s_{a} = 4\) & \(s_{a} = 5\) & \(s_{a} = 6\)\\
\hline
\(s_{a} = 1\) & 0 & 0 & 0 & 0 & 0 & 0\\
\(s_{a} = 2\) & 0 & 0 & 0 & 0 & 0 & 0\\
\(s_{a} = 3\) & 0 & 0 & 0 & 0 & 0 & 1/4\\
\(s_{a} = 4\) & 0 & 0 & 0 & 0 & 1/4 & 0\\
\(s_{a} = 5\) & 0 & 0 & 0 & 1/4 & 0 & 0\\
\(s_{a} = 6\) & 0 & 0 & 1/4 & 0 & 0 & 0\\
\hline
\end{tabular}
\end{table}
\end{instance}
\end{document}
