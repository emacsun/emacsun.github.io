% Intended LaTeX compiler: pdflatex
\documentclass[10pt,a4paper,UTF8]{article}
\usepackage{zclorg}
\author{张朝龙}
\date{}
\title{学习Python Doc第二天:控制语句和函数}
\hypersetup{
 pdfauthor={张朝龙},
 pdftitle={学习Python Doc第二天:控制语句和函数},
 pdfkeywords={},
 pdfsubject={},
 pdfcreator={Emacs 25.0.50.1 (Org mode 9.0.5)}, 
 pdflang={English}}
\begin{document}

\maketitle
\tableofcontents

今天是学习Python的第二天,主要内容涉及控制语句,函数定义以及代码风格。


\section{控制语句}
\label{sec:org50c70ff}


除了之前学习过的 \texttt{while} ,典型的控制语句还有 \texttt{if,else} \texttt{for,break,continue} 

\section{\texttt{if} 语句}
\label{sec:orgb442611}

看代码:
\lstset{language=Python,label= ,caption= ,captionpos=b,numbers=none}
\begin{lstlisting}
x = 23
if x<0:
    x = 0
    print('Negative changed to zero')
elif x==0:
    print('zero')
elif x == 1:
    print('Single')
else:
    print('More')
\end{lstlisting}
可以有0个 \texttt{elif} ,也可以有多个 \texttt{elif} ,最后的 =else= 也是可选的。
\section{\texttt{for} 语句}
\label{sec:org1e52e1f}


Python 的 \texttt{for} 语句和C语言中的 \texttt{for} 稍有不同。Python的 \texttt{for} 语句可以在任何 \texttt{list} 或者 \texttt{string} 之间循环。看代码:

\lstset{language=Python,label= ,caption= ,captionpos=b,numbers=none}
\begin{lstlisting}
words = ['cat','window','defenestrate']
for w in words:
    print(w,len(w))
\end{lstlisting}
输出为:
\begin{verbatim}
cat 3
window 6
defenestrate 12
\end{verbatim}

如果你需要在循环的过程中改变循环对象,建议首先创建一份寻欢对象的副本,看代码:
\lstset{language=Python,label= ,caption= ,captionpos=b,numbers=none}
\begin{lstlisting}
words = ['cat','window','defenestrate']
for w in words[:]:
    if len(w) > 6:
        words.insert(0,w)
\end{lstlisting}
输出
\begin{verbatim}
['defenestrate', 'cat', 'window', 'defenestrate']
\end{verbatim}
从代码中可以看出,在 \texttt{for w in words[:]:} 这一样代码中,Python隐含的创建了 \texttt{words[:]} 的一份copy。
\section{\texttt{range()} 函数}
\label{sec:org5ec42e7}


如果你需要在一个整数数组里循环, \texttt{range()} 函数使其易如反掌,看代码:
\lstset{language=Python,label= ,caption= ,captionpos=b,numbers=none}
\begin{lstlisting}
for i in range(5):
    print(i)
\end{lstlisting}

输出:
\begin{verbatim}
0
1
2
3
4
\end{verbatim}
可以看到, \texttt{range(n)} 返回的是 \texttt{[0,n-1]} .  \texttt{range()}  函数非常灵活,比如:
\begin{enumerate}
\item \texttt{range(5,9)} 生成从5到9,包含5不包含9的整数列表
\item \texttt{range(0,10,3)} 生成从0到10,包含0不包含10,间隔为3的整数列表
\item \texttt{range(-10,-100,-30)} 生成从-10到-100,间隔30的整数列表
\end{enumerate}

为了索引字符串构成的list,我们可以结合 \texttt{range()} 和 \texttt{len()} ,看代码:
\lstset{language=Python,label= ,caption= ,captionpos=b,numbers=none}
\begin{lstlisting}
a = ['Mary','had','a' ,'little','lamb']
for i in range(len(a)):
    print(i,a[i])
\end{lstlisting}
输出:
\begin{verbatim}
0 Mary
1 had
2 a
3 little
4 lamb
\end{verbatim}

山外有山,人外有人,函数外有函数,大多数情况下 \texttt{enumerate()} 是更方便的选择。关于 \texttt{enumerate()} 就不展开了。

关于 \texttt{range()} 有个非常奇怪的现象,大多数时候 \texttt{range()} 表现的像个 list,但是它不是。 \texttt{range()} 的返回是一个对象,这个对象可以用来迭代(这样的对象叫做迭代器,这个迭代器并不占用空间存储内容,因此比较节省空间)。 \texttt{list()} 函数也是迭代器( \texttt{iterator} ) .
\section{循环中的 \texttt{break} \texttt{continue} 和 \texttt{else}}
\label{sec:org60bf851}


\texttt{break} \texttt{continue} 的表象和 C 语言中类似。不同的是 \texttt{for} , \texttt{for} 可以有一个 \texttt{else} 。这个 \texttt{else} 在 \texttt{for} 循环结束后执行( \texttt{break} 跳出除外, \texttt{break} 跳出不会执行 \texttt{for} 的 \texttt{else} )。 看代码:

\lstset{language=Python,label= ,caption= ,captionpos=b,numbers=none}
\begin{lstlisting}
for n in range(2,10):
    for x in range(2,n):
        if n%x == 0:
            print(n, 'equals', x ,'*', n//x)
            break
    else:
        # loop fell through without finding a factor
        print(n,'is a prime number')
\end{lstlisting}

输出为:
\begin{verbatim}
2 is a prime number
3 is a prime number
4 equals 2 * 2
5 is a prime number
6 equals 2 * 3
7 is a prime number
8 equals 2 * 4
9 equals 3 * 3
\end{verbatim}

请再次注意: \texttt{else= 属于 =for} 不属于 \texttt{if} .

\texttt{continue} 和C中的 \texttt{continue} 一样, 看代码:
\lstset{language=Python,label= ,caption= ,captionpos=b,numbers=none}
\begin{lstlisting}
for num in range(2,10):
    if n%2 == 0:
        print('find an even number',num)
        continue
    print('find a number',num)
\end{lstlisting}

输出为:
\begin{verbatim}
find an even number 2
find a number 3
find an even number 4
find a number 5
find an even number 6
find a number 7
find an even number 8
find a number 9
\end{verbatim}
\section{什么也不做的 \texttt{pass}}
\label{sec:orgf569f06}


\texttt{pass} 语句什么也不做,常用来做语法占位,比如:
\lstset{language=Python,label= ,caption= ,captionpos=b,numbers=none}
\begin{lstlisting}
while True:
    pass
\end{lstlisting}

或者,创建一个最小的 \texttt{class} 
\lstset{language=Python,label= ,caption= ,captionpos=b,numbers=none}
\begin{lstlisting}
class MyEmptyClass:
    pass
\end{lstlisting}
另外 \texttt{pass} 也可以用来占位,稍后再回来实现相关功能。 比如:
\lstset{language=Python,label= ,caption= ,captionpos=b,numbers=none}
\begin{lstlisting}
def initlog(*args):
    pass % Remember to come back and fill the pass
\end{lstlisting}
\section{定义函数}
\label{sec:orge7c490f}


在Python中使用 \texttt{def} 来定义函数,看代码:
\lstset{language=Python,label= ,caption= ,captionpos=b,numbers=none}
\begin{lstlisting}
def fib(n):
    """Pring a Fibonacci series up to n"""
    a,b = 0,1
    while a<n:
        print(a,end=' ')
        a,b = b,a+b
    print()
\end{lstlisting}

调用 \texttt{fib(2000)} 输出:
\begin{verbatim}
0 1 1 2 3 5 8 13 21 34 55 89 144 233 377 610 987 1597
\end{verbatim}

函数的第一行往往是这个函数的文档,一句话说明这个函数的功能。Python有相应的工具可以把用这些注释生成文档供用户查看。执行函数过程中,会为函数中的变量生成一个本地符号表。系统检索变量值的顺序是:1. 本地符号表; 2. 全局符号表;3. 内置变量符号表。所以全局变量不能在一个函数中直接赋值(因为函数内创建的是本地符号表),但是全局变量在函数中确实可以被引用的。

一个函数的定义在当前符号表中引入了这个函数的名字。解释器把函数名字识别为用户定义的函数。这个值可以赋给另外的名字。看代码:
\lstset{language=Python,label= ,caption= ,captionpos=b,numbers=none}
\begin{lstlisting}
def fib(n):
    """Pring a Fibonacci series up to n"""
    a,b = 0,1
    while a<n:
        print(a,end=' ')
        a,b = b,a+b
    print()

f = fib
f(1000)
\end{lstlisting}

输出为:
\begin{verbatim}
0 1 1 2 3 5 8 13 21 34 55 89 144 233 377 610 987
\end{verbatim}

如果有其他计算机语言的基础,你会疑惑: \texttt{fib} 只能是一个代码段,不能算一个函数,因为根本没有返回值。事实上,即使没有 \texttt{return} 语句的函数也有返回值,只不过这个返回值无关紧要。这个值是 \texttt{None} 。如果你一定要看到这个 \texttt{None} ,看代码:
\begin{verbatim}
>>>print(fib(0))

None
\end{verbatim}

写一个返回菲波那切数列的函数很容易,看代码:
\lstset{language=Python,label= ,caption= ,captionpos=b,numbers=none}
\begin{lstlisting}
def fib(n):
    """Pring a Fibonacci series up to n"""
    result = []
    a,b = 0,1
    while a<n:
        result.append(a)
        a,b = b,a+b
    return result
\end{lstlisting}
调用:
\lstset{language=Python,label= ,caption= ,captionpos=b,numbers=none}
\begin{lstlisting}
>>> f100 = fib(100)
>>> f100
[0, 1, 1, 2, 3, 5, 8, 13, 21, 34, 55, 89]
\end{lstlisting}

从这个例子,我们得知:
\begin{enumerate}
\item \texttt{return} 语句返回一个值。如果没有 \texttt{return} 语句,返回 \texttt{None} 。
\item \texttt{result.append(a)}  调用了 \texttt{result} 对象的一个方法 \texttt{append} 。不同类型的对象有不同的方法,尽管这些方法可能名字相同,但是由于类型不同,所以在使用过程中也不会造成歧义。
\end{enumerate}
\end{document}
