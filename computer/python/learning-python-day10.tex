% Intended LaTeX compiler: pdflatex
\documentclass[10pt,a4paper,UTF8]{article}
\usepackage{zclorg}
\date{}
\title{学习Python Doc第十天: 标准库巡礼(二)}
\hypersetup{
 pdfauthor={},
 pdftitle={学习Python Doc第十天: 标准库巡礼(二)},
 pdfkeywords={},
 pdfsubject={},
 pdfcreator={Emacs 25.0.50.1 (Org mode 9.0.5)},
 pdflang={English}}
\begin{document}

\maketitle
\tableofcontents
\titlepic{\includegraphics[scale=0.25]{../../img/sinc.PNG}}
之前对标准库中的部分模块已经有了简单的介绍,\href{learning-python-day09.org}{见这里} 。今天,我们浏览剩下的部分。我只快速浏览,详细的用法留到工程实践中去。所以这部分内容大概可以五分钟过完。

\section{输出格式}
\label{sec:org1c963f1}


\texttt{reprlib} 模块提供了一个 \texttt{repr()} 的版本,用于显示较大的容器的简写。

\begin{verbatim}
>>> import reprlib
>>> reprlib.repr(set('supercalifragilisticexpialidocious'))
"{'a', 'c', 'd', 'e', 'f', 'g', ...}"
\end{verbatim}

在输出内置的或者用户定义的对象时, \texttt{pprint} 模块提供了更精巧的控制。 当结果冲过一行时, \texttt{pretty printer} 会自动断行,并美化显示结果。

\texttt{textwrap} 模块会格式化一段文字,使其适合在给定的屏幕上显示。
\section{模板}
\label{sec:org6fc2d75}


\texttt{string} 模块提供了 \texttt{Template} 类。这个类的语法简单,适合普通用户使用。格式字符创中使用 \$ 定位 \texttt{Pytone} 变量位置。看代码:
\begin{verbatim}
>>> from string import Template
>>> t = Template('${village}folk send $$10 to $cause.')
>>> t.substitute(village='Nottingham', cause='the ditch fund')
'Nottinghamfolk send $10 to the ditch fund.'
\end{verbatim}


当一个\$ 后的变量没有被替换时, \texttt{substitude()} 方法会产生一个 \texttt{KeyError} 的错误。

\section{二进制文件工具}
\label{sec:orged15108}


\texttt{struct} 模块提供了 \texttt{pack()} 和 \texttt{unpack()} 函数。这两个函数可以用来操作二进制文件。
\begin{verbatim}
import struct

with open('myfile.zip', 'rb') as f:
    data = f.read()

start = 0
for i in range(3):                      # show the first 3 file headers
    start += 14
    fields = struct.unpack('<IIIHH', data[start:start+16])
    crc32, comp_size, uncomp_size, filenamesize, extra_size = fields

    start += 16
    filename = data[start:start+filenamesize]
    start += filenamesize
    extra = data[start:start+extra_size]
    print(filename, hex(crc32), comp_size, uncomp_size)

    start += extra_size + comp_size     # skip to the next header
\end{verbatim}
\section{多线程}
\label{sec:org8cdcb6d}


多线程使得并行计算成为可能。 \texttt{threading} 模块提供了很多函数用于产生多线程。

\begin{verbatim}
import threading, zipfile

class AsyncZip(threading.Thread):
    def __init__(self, infile, outfile):
        threading.Thread.__init__(self)
        self.infile = infile
        self.outfile = outfile

    def run(self):
        f = zipfile.ZipFile(self.outfile, 'w', zipfile.ZIP_DEFLATED)
        f.write(self.infile)
        f.close()
        print('Finished background zip of:', self.infile)

background = AsyncZip('mydata.txt', 'myarchive.zip')
background.start()
print('The main program continues to run in foreground.')

background.join()    # Wait for the background task to finish
print('Main program waited until background was done.')
\end{verbatim}
多线程编程的最大挑战是协调多个线程的数据和其他计算资源。 \texttt{threading} 模块提供了很多同步机制用来保证数据一致性,这些同步机制包括: \texttt{locks,events,condition variables, semaphores}
\section{日志}
\label{sec:org045fe20}


\texttt{logging} 模块提供了全能且灵活的日志系统。最简单的情况是:用文件或者 \texttt{sys.stderr} 来记录日志。

\begin{verbatim}
import logging
logging.debug('Debugging information')
logging.info('Informational message')
logging.warning('Warning:config file %s not found', 'server.conf')
logging.error('Error occurred')
logging.critical('Critical error -- shutting down')
\end{verbatim}

输出为:
\begin{verbatim}
WARNING:root:Warning:config file server.conf not found
ERROR:root:Error occurred
CRITICAL:root:Critical error -- shutting down
\end{verbatim}

\section{弱引用}
\label{sec:orge823e8e}


\texttt{Python} 提供自动内存管理机制,有自己的 \texttt{garbage collection} 系统。对大多数对象执行 \texttt{reference counting} 。当最后一个reference消失的时候,对象的内存被释放。

这套机制在大多数应用中都工作良好。但是,偶尔情况下,我们需要对某个对象进行长时间的追踪。不幸的是,紧紧追踪这些对象会产生一个永远也不能消除reference。 \texttt{weakref} 模块提供了追踪对象而不生成reference的方法。当一个对象不再使用,它会从弱引用表中删除。
\end{document}
